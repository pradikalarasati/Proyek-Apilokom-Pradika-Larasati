\documentclass{article}

\usepackage{eumat}

\begin{document}
\begin{eulernotebook}
\eulerheading{EMT untuk Perhitungan Aljabar}
\begin{eulercomment}
Pada notebook ini Anda belajar menggunakan EMT untuk melakukan
berbagai perhitungan terkait dengan materi atau topik dalam Aljabar.
Kegiatan yang harus Anda lakukan adalah sebagai berikut:

- Membaca secara cermat dan teliti notebook ini;\\
- Menerjemahkan teks bahasa Inggris ke bahasa Indonesia;\\
- Mencoba contoh-contoh perhitungan (perintah EMT) dengan cara
meng-ENTER setiap perintah EMT yang ada (pindahkan kursor ke baris
perintah)\\
- Jika perlu Anda dapat memodifikasi perintah yang ada dan memberikan
keterangan/penjelasan tambahan terkait hasilnya.\\
- Menyisipkan baris-baris perintah baru untuk mengerjakan soal-soal
Aljabar dari file PDF yang saya berikan;\\
- Memberi catatan hasilnya.\\
- Jika perlu tuliskan soalnya pada teks notebook (menggunakan format
LaTeX).\\
- Gunakan tampilan hasil semua perhitungan yang eksak atau simbolik
dengan format LaTeX. (Seperti contoh-contoh pada notebook ini.)

\end{eulercomment}
\eulersubheading{Contoh pertama}
\begin{eulercomment}
==Sisipan Soal dari PDF \textbar{} sub topik : operasi bentuk-bentuk aljabar==

Menyederhanakan bentuk aljabar:

\end{eulercomment}
\begin{eulerformula}
\[
6x^{-3}y^5\times -7x^2y^{-9}
\]
\end{eulerformula}
\begin{eulercomment}
\end{eulercomment}
\begin{eulerprompt}
>$&6*x^(-3)*y^5*-7*x^2*y^(-9)
\end{eulerprompt}
\begin{eulerformula}
\[
-\frac{42}{x\,y^4}
\]
\end{eulerformula}
\begin{eulercomment}
Bentuk paling sederhananya adalah -(42/xy\textasciicircum{}4)

Soal Kedua:\\
\end{eulercomment}
\begin{eulerformula}
\[
\left((\frac{x^r}{y^t})^2(\frac{x^{2r}}{y^{4t}})^{-2}\right)^{-3}
\]
\end{eulerformula}
\begin{eulerprompt}
>$&((x^r/y^t)^2*(x^(2*r)/y^(4*t))^(-2))^(-3)
\end{eulerprompt}
\begin{eulerformula}
\[
\frac{x^{6\,r}}{y^{18\,t}}
\]
\end{eulerformula}
\begin{eulercomment}
Bentuk paling sederhana dari soal kedua adalah x\textasciicircum{}(6r)/y\textasciicircum{}(18t)

Soal Ketiga:\\
\end{eulercomment}
\begin{eulerformula}
\[
\left(\frac{(3x^ay^b)^3}{(-3x^ay^b)^2}\right)^2
\]
\end{eulerformula}
\begin{eulerprompt}
>$&((3*x^a*y^b)^3/(-3*x^a*y^b)^2)^2
\end{eulerprompt}
\begin{eulerformula}
\[
9\,x^{2\,a}\,y^{2\,b}
\]
\end{eulerformula}
\begin{eulercomment}
Bentuk paling sederhana dari soal ketiga adalah 9x\textasciicircum{}(2a)y\textasciicircum{}(2b)
\end{eulercomment}
\begin{eulercomment}

Menjabarkan bentuk aljabar

Soal Pertama:\\
\end{eulercomment}
\begin{eulerformula}
\[
(6x^{-3}+y^5)(-7x^2-y^{-9})
\]
\end{eulerformula}
\begin{eulerprompt}
>$&showev('expand((6*x^(-3)+y^5)*(-7*x^2-y^(-9))))
\end{eulerprompt}
\begin{eulerformula}
\[
{\it expand}\left(\left(-\frac{1}{y^9}-7\,x^2\right)\,\left(y^5+  \frac{6}{x^3}\right)\right)=-7\,x^2\,y^5-\frac{1}{y^4}-\frac{6}{x^3  \,y^9}-\frac{42}{x}
\]
\end{eulerformula}
\begin{eulercomment}
Soal Kedua:\\
\end{eulercomment}
\begin{eulerformula}
\[
(y-5)^2
\]
\end{eulerformula}
\begin{eulerprompt}
>$&showev('expand((y-5)^2))
\end{eulerprompt}
\begin{eulerformula}
\[
{\it expand}\left(\left(y-5\right)^2\right)=y^2-10\,y+25
\]
\end{eulerformula}
\begin{eulercomment}
penjabaran dari bentuk kuadrat (y-5)\textasciicircum{}2 adalah y\textasciicircum{}2-10y-25


Memfaktorkan bentuk aljabar

Soal Pertama:\\
\end{eulercomment}
\begin{eulerformula}
\[
9z^2-24z+16
\]
\end{eulerformula}
\begin{eulerprompt}
>$&factor(9*z^2-24*z+16)
\end{eulerprompt}
\begin{eulerformula}
\[
\left(3\,z-4\right)^2
\]
\end{eulerformula}
\begin{eulercomment}
hasil faktor dari p\textasciicircum{}3-2p\textasciicircum{}2-9p+18 adalah (p-3)(p-2)(p+3)

Soal Ketiga:\\
\end{eulercomment}
\begin{eulerformula}
\[
p-64p^4
\]
\end{eulerformula}
\begin{eulerprompt}
>$&factor(p-64*p^4)
\end{eulerprompt}
\begin{eulerformula}
\[
-p\,\left(4\,p-1\right)\,\left(16\,p^2+4\,p+1\right)
\]
\end{eulerformula}
\begin{eulercomment}
faktor yang dihasilkan adalah (-p)(4p-1)(16p\textasciicircum{}2+4p+1)

Saya menambahkan 5 soal pada bagian ini yaitu menambahkan 3 soal
memfaktorkan bentuk aljabar, 1 soal menjabarkan bentuk aljabar, dan 1
soal menyederhanakan bentuk aljabar.


\end{eulercomment}
\eulersubheading{Baris Perintah}
\begin{eulercomment}
Baris perintah Euler terdiri dari satu atau beberapa perintah Euler
yang diikuti dengan titik koma ";" atau koma ",". Titik koma mencegah
pencetakan hasil. Koma setelah perintah terakhir dapat dihilangkan.

Baris perintah berikut ini hanya akan mencetak hasil dari ekspresi,
bukan penugasan atau perintah format.
\end{eulercomment}
\begin{eulerprompt}
>r:=2; h:=4; pi*r^2*h/3
\end{eulerprompt}
\begin{euleroutput}
  16.7551608191
\end{euleroutput}
\begin{eulercomment}
Perintah harus dipisahkan dengan tanda kosong. Baris perintah berikut
ini mencetak dua hasilnya.
\end{eulercomment}
\begin{eulerprompt}
>pi*2*r*h, %+2*pi*r*h // Ingat tanda % menyatakan hasil perhitungan terakhir sebelumnya
\end{eulerprompt}
\begin{euleroutput}
  50.2654824574
  100.530964915
\end{euleroutput}
\begin{eulercomment}
Baris perintah dieksekusi sesuai urutan pengguna menekan tombol
return. Jadi, Anda mendapatkan nilai baru setiap kali Anda
mengeksekusi baris kedua.
\end{eulercomment}
\begin{eulerprompt}
>x := 1;
>x := cos(x) // nilai cosinus (x dalam radian)
\end{eulerprompt}
\begin{euleroutput}
  0.540302305868
\end{euleroutput}
\begin{eulerprompt}
>x := cos(x)
\end{eulerprompt}
\begin{euleroutput}
  0.857553215846
\end{euleroutput}
\begin{eulercomment}
Jika dua baris dihubungkan dengan "...", kedua baris tersebut akan
selalu dieksekusi secara bersamaan.
\end{eulercomment}
\begin{eulerprompt}
>x := 1.5; ...
>x := (x+2/x)/2, x := (x+2/x)/2, x := (x+2/x)/2, 
\end{eulerprompt}
\begin{euleroutput}
  1.41666666667
  1.41421568627
  1.41421356237
\end{euleroutput}
\begin{eulercomment}
Ini juga merupakan cara yang baik untuk membagi perintah yang panjang
menjadi dua baris atau lebih. Anda dapat menekan Ctrl+Return untuk
membagi baris menjadi dua pada posisi kursor saat ini, atau Ctlr+Back
untuk menggabungkan kedua baris.

Untuk melipat semua multi-baris, tekan Ctrl+L. Kemudian garis-garis
berikutnya hanya akan terlihat, jika salah satu dari mereka memiliki
fokus. Untuk melipat satu baris multi-baris, mulai baris pertama
dengan "\%+ ".
\end{eulercomment}
\begin{eulerprompt}
>%+ x=4+5; ...
\end{eulerprompt}
\begin{eulercomment}
Garis yang dimulai dengan \%\% tidak akan terlihat sama sekali.
\end{eulercomment}
\begin{euleroutput}
  81
\end{euleroutput}
\begin{eulercomment}
Euler mendukung perulangan dalam baris perintah, selama perulangan
tersebut masuk ke dalam satu baris tunggal atau beberapa baris. Dalam
program, tentu saja pembatasan ini tidak berlaku. Untuk informasi
lebih lanjut, baca pengantar berikut ini.
\end{eulercomment}
\begin{eulerprompt}
>x=1; for i=1 to 5; x := (x+2/x)/2, end; // menghitung akar 2
\end{eulerprompt}
\begin{euleroutput}
  1.5
  1.41666666667
  1.41421568627
  1.41421356237
  1.41421356237
\end{euleroutput}
\begin{eulercomment}
Tidak masalah untuk menggunakan multi-baris. Pastikan baris diakhiri
dengan "...".
\end{eulercomment}
\begin{eulerprompt}
>x := 1.5; // comments go here before the ...
>repeat xnew:=(x+2/x)/2; until xnew~=x; ...
>   x := xnew; ...
>end; ...
>x,
\end{eulerprompt}
\begin{euleroutput}
  1.41421356237
\end{euleroutput}
\begin{eulercomment}
Struktur bersyarat juga bisa digunakan.
\end{eulercomment}
\begin{eulerprompt}
>if E^pi>pi^E; then "Thought so!", endif;
\end{eulerprompt}
\begin{euleroutput}
  Thought so!
\end{euleroutput}
\begin{eulercomment}
Ketika Anda menjalankan perintah, kursor dapat berada di posisi mana
pun dalam baris perintah. Anda dapat kembali ke perintah sebelumnya
atau melompat ke perintah berikutnya dengan tombol panah. Atau Anda
dapat mengklik bagian komentar di atas perintah untuk membuka perintah
tersebut.

Ketika Anda menggerakkan kursor di sepanjang baris, pasangan tanda
kurung atau tanda kurung pembuka dan penutup akan disorot. Juga,
perhatikan baris status. Setelah tanda kurung pembuka dari fungsi
sqrt(), baris status akan menampilkan teks bantuan untuk fungsi
tersebut. Jalankan perintah dengan tombol return.
\end{eulercomment}
\begin{eulerprompt}
>sqrt(sin(10°)/cos(20°))
\end{eulerprompt}
\begin{euleroutput}
  0.429875017772
\end{euleroutput}
\begin{eulercomment}
Untuk melihat bantuan untuk perintah terbaru, buka jendela bantuan
dengan F1. Di sana, Anda dapat memasukkan teks yang akan dicari. Pada
baris kosong, bantuan untuk jendela bantuan akan ditampilkan. Anda
dapat menekan escape untuk mengosongkan baris, atau menutup jendela
bantuan.

Anda dapat mengklik dua kali pada perintah apa pun untuk membuka
bantuan untuk perintah ini. Coba klik dua kali perintah exp di bawah
ini pada baris perintah.
\end{eulercomment}
\begin{eulerprompt}
>exp(log(2.5))
\end{eulerprompt}
\begin{euleroutput}
  2.5
\end{euleroutput}
\begin{eulercomment}
Anda juga dapat menyalin dan menempel di Euler. Gunakan Ctrl-C dan
Ctrl-V untuk ini. Untuk menandai teks, seret mouse atau gunakan shift
bersamaan dengan tombol kursor. Selain itu, Anda dapat menyalin tanda
kurung yang disorot.
\end{eulercomment}
\begin{eulercomment}


\end{eulercomment}
\eulersubheading{Sintaksis Dasar}
\begin{eulercomment}
Euler mengetahui fungsi matematika yang biasa. Seperti yang telah Anda
lihat di atas, fungsi trigonometri bekerja dalam radian atau derajat.
Untuk mengonversi ke derajat, tambahkan simbol derajat (dengan tombol
F7) ke nilai, atau gunakan fungsi rad(x). Fungsi akar kuadrat disebut
sqrt dalam Euler. Tentu saja, x\textasciicircum{}(1/2) juga dapat digunakan.

Untuk mengatur variabel, gunakan "=" atau ":=". Demi kejelasan,
pengantar ini menggunakan bentuk yang terakhir. Spasi tidak menjadi
masalah. Tetapi spasi antar perintah diharapkan.

Beberapa perintah dalam satu baris dipisahkan dengan "," atau ";".
Titik koma menekan output dari perintah. Pada akhir baris perintah,
"," diasumsikan, jika ";" tidak ada.
\end{eulercomment}
\begin{eulerprompt}
>g:=9.81; t:=2.5; 1/2*g*t^2
\end{eulerprompt}
\begin{euleroutput}
  30.65625
\end{euleroutput}
\begin{eulercomment}
EMT menggunakan sintaks pemrograman untuk ekspresi. Untuk memasukkan

\end{eulercomment}
\begin{eulerformula}
\[
e^2 \cdot \left( \frac{1}{3+4 \log(0.6)}+\frac{1}{7} \right)
\]
\end{eulerformula}
\begin{eulercomment}
Anda harus mengatur tanda kurung yang benar dan menggunakan / untuk
pecahan. Perhatikan tanda kurung yang disorot untuk mendapatkan
bantuan. Perhatikan bahwa konstanta Euler e diberi nama E dalam EMT.
\end{eulercomment}
\begin{eulerprompt}
>E^2*(1/(3+4*log(0.6))+1/7)
\end{eulerprompt}
\begin{euleroutput}
  8.77908249441
\end{euleroutput}
\begin{eulercomment}
Untuk menghitung ekspresi yang rumit seperti

\end{eulercomment}
\begin{eulerformula}
\[
\left(\frac{\frac17 + \frac18 + 2}{\frac13 + \frac12}\right)^2 \pi
\]
\end{eulerformula}
\begin{eulercomment}
Anda harus memasukkannya dalam bentuk baris.
\end{eulercomment}
\begin{eulerprompt}
>((1/7 + 1/8 + 2) / (1/3 + 1/2))^2 * pi
\end{eulerprompt}
\begin{euleroutput}
  23.2671801626
\end{euleroutput}
\begin{eulercomment}
Letakkan tanda kurung di sekitar sub-ekspresi yang perlu dihitung
terlebih dahulu. EMT membantu Anda dengan menyorot ekspresi yang
diselesaikan oleh tanda kurung penutup. Anda juga harus memasukkan
nama "pi" untuk huruf Yunani pi.

Hasil dari perhitungan ini adalah angka floating point. Secara default
dicetak dengan akurasi sekitar 12 digit. Pada baris perintah berikut,
kita juga belajar bagaimana kita dapat merujuk ke hasil sebelumnya
dalam baris yang sama.
\end{eulercomment}
\begin{eulerprompt}
>1/3+1/7, fraction %
\end{eulerprompt}
\begin{euleroutput}
  0.47619047619
  10/21
\end{euleroutput}
\begin{eulercomment}
Perintah Euler dapat berupa ekspresi atau perintah primitif. Ekspresi
terbuat dari operator dan fungsi. Jika perlu, ekspresi tersebut harus
mengandung tanda kurung untuk memaksa urutan eksekusi yang benar. Jika
ragu, mengatur tanda kurung adalah ide yang bagus. Perhatikan bahwa
EMT menampilkan tanda kurung pembuka dan penutup saat mengedit baris
perintah.
\end{eulercomment}
\begin{eulerprompt}
>(cos(pi/4)+1)^3*(sin(pi/4)+1)^2
\end{eulerprompt}
\begin{euleroutput}
  14.4978445072
\end{euleroutput}
\begin{eulercomment}
Operator numerik Euler meliputi

\end{eulercomment}
\begin{eulerttcomment}
 + unary atau operator plus
 - unary atau operator minus
 *, /
 . produk matriks
 pangkat a^b untuk a positif atau bilangan bulat b (a**b juga bisa
\end{eulerttcomment}
\begin{eulercomment}
digunakan)\\
\end{eulercomment}
\begin{eulerttcomment}
 n! operator faktorial
\end{eulerttcomment}
\begin{eulercomment}

dan masih banyak lagi.

Berikut adalah beberapa fungsi yang mungkin Anda perlukan. Masih
banyak lagi.

\end{eulercomment}
\begin{eulerttcomment}
 sin, cos, tan, atan, asin, acos, rad, deg
 log, exp, log10, sqrt, logbase
 bin, logbin, logfac, mod, floor, ceil, round, abs, sign
 conj,re,im,arg,conj,real,complex
 beta,betai,gamma,complexgamma,ellrf,ellf,ellrd,elle
 bitand, bitor, bitxor, bitnot
\end{eulerttcomment}
\begin{eulercomment}

Beberapa perintah memiliki alias, misalnya ln untuk log.
\end{eulercomment}
\begin{eulerprompt}
>ln(E^2), arctan(tan(0.5))
\end{eulerprompt}
\begin{euleroutput}
  2
  0.5
\end{euleroutput}
\begin{eulerprompt}
>sin(30°)
\end{eulerprompt}
\begin{euleroutput}
  0.5
\end{euleroutput}
\begin{eulercomment}
Pastikan untuk menggunakan tanda kurung (tanda kurung bulat), apabila
ada keraguan tentang urutan eksekusi! Berikut ini tidak sama dengan
(2\textasciicircum{}3)\textasciicircum{}4, yang merupakan default untuk 2\textasciicircum{}3\textasciicircum{}4 di EMT (beberapa sistem
numerik melakukannya dengan cara lain).
\end{eulercomment}
\begin{eulerprompt}
>2^3^4, (2^3)^4, 2^(3^4)
\end{eulerprompt}
\begin{euleroutput}
  2.41785163923e+24
  4096
  2.41785163923e+24
\end{euleroutput}
\begin{eulercomment}
==Sisipan Soal dari PDF \textbar{} sub topik : operasi \& fungsi matematika==

1) berapakah hasil dari pengurangan berikut?

\end{eulercomment}
\begin{eulerformula}
\[
(5x^2+4xy-3y^2+2)-(9x^2-4xy+2y^2-1)
\]
\end{eulerformula}
\begin{eulerprompt}
>$&(5*x^2+4*x*y-3*y^2+2)-(9*x^2-4*x*y+2*y^2-1)
\end{eulerprompt}
\begin{eulerformula}
\[
-5\,y^2+8\,x\,y-4\,x^2+3
\]
\end{eulerformula}
\begin{eulercomment}
Hasil dari pengurangan tersebut adalah -5y\textasciicircum{}2+8xy-4x\textasciicircum{}2+3
\end{eulercomment}
\begin{eulercomment}
2) Berapakah hasil pembagian dari\\
\end{eulercomment}
\begin{eulerformula}
\[
\frac{2x^4+3x^2-1}{x-\frac{1}{2}}
\]
\end{eulerformula}
\begin{eulerprompt}
>$&(2*x^4 + 3*x^2 - 1) / (x - 1/2)
\end{eulerprompt}
\begin{eulerformula}
\[
\frac{2\,x^4+3\,x^2-1}{x-\frac{1}{2}}
\]
\end{eulerformula}
\begin{euleroutput}
  3) berapakah hasil dari f(x) dikali g(x)?
  
  latex: f(x) = 3a^2,\(\backslash\)quad g(x)=(-7a^4)
\end{euleroutput}
\begin{eulercomment}
Hasil dari pembagian dari soal tersebut adalah

3) Berapakah hasil dari f(x) dikali g(x)?

\end{eulercomment}
\begin{eulerformula}
\[
f(x) = 3a^2,\quad g(x)=(-7a^4)
\]
\end{eulerformula}
\begin{eulerprompt}
>$&(3*a^2)*(-7*a^4)
\end{eulerprompt}
\begin{eulerformula}
\[
-21\,a^6
\]
\end{eulerformula}
\begin{eulercomment}
hasil dari f(x) dikali dengan g(x) adalah -21a\textasciicircum{}6

4) berapakah hasil dari operasi berikut?

\end{eulercomment}
\begin{eulerformula}
\[
(x+3)^2
\]
\end{eulerformula}
\begin{eulerprompt}
>$&expand((x+3)^2)
\end{eulerprompt}
\begin{eulerformula}
\[
x^2+6\,x+9
\]
\end{eulerformula}
\begin{eulercomment}
Hasil dari operasi tersebut adalah x\textasciicircum{}2+6x+9

5)berapakah hasil dari penjumlahan berikut?

\end{eulercomment}
\begin{eulerformula}
\[
(2x^2+12xy-11)+(6x^2-2x+4)+(-x^2-y-2)
\]
\end{eulerformula}
\begin{eulerprompt}
>$&(2*x^2+12*x*y-11)+(6*x^2-2*x+4)+(-x^2-y-2)
\end{eulerprompt}
\begin{eulerformula}
\[
12\,x\,y-y+7\,x^2-2\,x-9
\]
\end{eulerformula}
\begin{eulercomment}
Hasil dari penjumlahan tersebut adalah 12xy-y+7x\textasciicircum{}2-2x-9



\end{eulercomment}
\eulersubheading{Bilangan Real}
\begin{eulercomment}
Tipe data utama dalam Euler adalah bilangan real. Bilangan real
direpresentasikan dalam format IEEE dengan akurasi sekitar 16 digit
desimal.
\end{eulercomment}
\begin{eulerprompt}
>longest 1/3
\end{eulerprompt}
\begin{euleroutput}
       0.3333333333333333 
\end{euleroutput}
\begin{eulercomment}
Representasi ganda internal membutuhkan 8 byte.
\end{eulercomment}
\begin{eulerprompt}
>printdual(1/3)
\end{eulerprompt}
\begin{euleroutput}
  1.0101010101010101010101010101010101010101010101010101*2^-2
\end{euleroutput}
\begin{eulerprompt}
>printhex(1/3)
\end{eulerprompt}
\begin{euleroutput}
  5.5555555555554*16^-1
\end{euleroutput}
\begin{eulercomment}
== Sisipan soal dari PDF \textbar{} Subtopik : perhitungan bilangan kompleks ==

Bilangan selain bilangan real terdapat juga bilangan kompleks.
Bilangan kompleks merupakan penjumlahan antara bilangan real dan
bilangan imajiner. berikut contoh-contoh perhitungan bilangan
kompleks:

1)berapakah nilai dari\\
\end{eulercomment}
\begin{eulerformula}
\[
(10+7i)-(5+3i)
\]
\end{eulerformula}
\begin{eulerprompt}
>$&(10+7*sqrt(-1))-(5+3*sqrt(-1))
\end{eulerprompt}
\begin{eulerformula}
\[
4\,i+5
\]
\end{eulerformula}
\begin{eulercomment}
nilai dari(10+7i)-(5+3i)adalah 4i+5

2) berapakah jumlah dari\\
\end{eulercomment}
\begin{eulerformula}
\[
(7-2i)+(4-5i)
\]
\end{eulerformula}
\begin{eulerprompt}
>$&(7-2*sqrt(-1))+(4-5*sqrt(-1))
\end{eulerprompt}
\begin{eulerformula}
\[
11-7\,i
\]
\end{eulerformula}
\begin{eulercomment}
Hasil dari operasi tersebut adalah 11-7i

3) berapakah hasil dari\\
\end{eulercomment}
\begin{eulerformula}
\[
\sqrt(-16)\cdot \sqrt(-100)
\]
\end{eulerformula}
\begin{eulerprompt}
>$&(sqrt(-16))*(sqrt(-100))
\end{eulerprompt}
\begin{eulerformula}
\[
-40
\]
\end{eulerformula}
\begin{eulercomment}
hasil dari operasi tersebut adalah -40

4) berapakah hasil operasi berikut?\\
\end{eulercomment}
\begin{eulerformula}
\[
-2i(-8+3i)
\]
\end{eulerformula}
\begin{eulerprompt}
>$&(-2*sqrt(-1))*(-8)+(-2*sqrt(-1))*(3*sqrt(-1))
\end{eulerprompt}
\begin{eulerformula}
\[
16\,i+6
\]
\end{eulerformula}
\begin{eulercomment}
hasil dari operasi tersebut adalah 16i+6

5) berapakah hasil bagi operasi berikut ini?\\
\end{eulercomment}
\begin{eulerformula}
\[
\frac{4-2i}{1+i}+\frac{2-5i}{1+i}
\]
\end{eulerformula}
\begin{eulerprompt}
>$&((4-2*sqrt(-1))+(2-5*sqrt(-1)))/(1+sqrt(-1))
\end{eulerprompt}
\begin{eulerformula}
\[
\frac{6-7\,i}{i+1}
\]
\end{eulerformula}
\begin{eulercomment}
hasil dari operasi tersebut adalah (6-7i)/(1+i)

\end{eulercomment}
\eulersubheading{String}
\begin{eulercomment}
String dalam Euler didefinisikan dengan "...".
\end{eulercomment}
\begin{eulerprompt}
>"A string can contain anything."
\end{eulerprompt}
\begin{euleroutput}
  A string can contain anything.
\end{euleroutput}
\begin{eulercomment}
String dapat digabungkan dengan \textbar{} atau dengan +. Ini juga berfungsi
dengan angka, yang dikonversi menjadi string dalam kasus tersebut.
\end{eulercomment}
\begin{eulerprompt}
>"The area of the circle with radius " + 2 + " cm is " + pi*4 + " cm^2."
\end{eulerprompt}
\begin{euleroutput}
  The area of the circle with radius 2 cm is 12.5663706144 cm^2.
\end{euleroutput}
\begin{eulercomment}
Fungsi cetak juga mengonversi angka ke string. Fungsi ini dapat
mengambil sejumlah digit dan sejumlah tempat (0 untuk output padat),
dan secara optimal satu unit.
\end{eulercomment}
\begin{eulerprompt}
>"Golden Ratio : " + print((1+sqrt(5))/2,5,0)
\end{eulerprompt}
\begin{euleroutput}
  Golden Ratio : 1.61803
\end{euleroutput}
\begin{eulercomment}
Ada string khusus tidak ada, yang tidak mencetak. Dikembalikan oleh
beberapa fungsi, ketika hasilnya tidak penting. (Dikembalikan secara
otomatis, jika fungsi tidak memiliki pernyataan pengembalian).
\end{eulercomment}
\begin{eulerprompt}
>none
\end{eulerprompt}
\begin{eulercomment}
Untuk mengonversi string menjadi angka, cukup evaluasi string
tersebut. Ini juga berlaku untuk ekspresi (lihat di bawah).
\end{eulercomment}
\begin{eulerprompt}
>"1234.5"()
\end{eulerprompt}
\begin{euleroutput}
  1234.5
\end{euleroutput}
\begin{eulercomment}
Untuk mendefinisikan vektor string, gunakan notasi vektor [...].
\end{eulercomment}
\begin{eulerprompt}
>v:=["affe","charlie","bravo"]
\end{eulerprompt}
\begin{euleroutput}
  affe
  charlie
  bravo
\end{euleroutput}
\begin{eulercomment}
Vektor string kosong dilambangkan dengan [none]. Vektor string dapat
digabungkan.
\end{eulercomment}
\begin{eulerprompt}
>w:=[none]; w|v|v
\end{eulerprompt}
\begin{euleroutput}
  affe
  charlie
  bravo
  affe
  charlie
  bravo
\end{euleroutput}
\begin{eulercomment}
String dapat berisi karakter Unicode. Secara internal, string ini
berisi kode UTF-8. Untuk membuat string seperti itu, gunakan u"..."
dan salah satu entitas HTML.

String Unicode dapat digabungkan seperti string lainnya.
\end{eulercomment}
\begin{eulerprompt}
>u"&alpha; = " + 45 + u"&deg;" // pdfLaTeX mungkin gagal menampilkan secara benar
\end{eulerprompt}
\begin{euleroutput}
  α = 45°
\end{euleroutput}
\begin{eulercomment}
I
\end{eulercomment}
\begin{eulercomment}
Di komentar, entitas yang sama seperti α, β dll. dapat
digunakan. Ini mungkin merupakan alternatif cepat untuk Lateks.
(Detail lebih lanjut di komentar di bawah).
\end{eulercomment}
\begin{eulercomment}
Ada beberapa fungsi untuk membuat atau menganalisis string unicode.
Fungsi strtochar() akan mengenali string Unicode, dan menerjemahkannya
dengan benar.
\end{eulercomment}
\begin{eulerprompt}
>v=strtochar(u"&Auml; is a German letter")
\end{eulerprompt}
\begin{euleroutput}
  [196,  32,  105,  115,  32,  97,  32,  71,  101,  114,  109,  97,  110,
  32,  108,  101,  116,  116,  101,  114]
\end{euleroutput}
\begin{eulercomment}
Hasilnya adalah sebuah vektor angka Unicode. Fungsi kebalikannya
adalah chartoutf().
\end{eulercomment}
\begin{eulerprompt}
>v[1]=strtochar(u"&Uuml;")[1]; chartoutf(v)
\end{eulerprompt}
\begin{euleroutput}
  Ü is a German letter
\end{euleroutput}
\begin{eulercomment}
Fungsi utf() dapat menerjemahkan sebuah string dengan entitas dalam
sebuah variabel menjadi sebuah string Unicode.
\end{eulercomment}
\begin{eulerprompt}
>s="We have &alpha;=&beta;."; utf(s) // pdfLaTeX mungkin gagal menampilkan secara benar
\end{eulerprompt}
\begin{euleroutput}
  We have α=β.
\end{euleroutput}
\begin{eulercomment}
Dimungkinkan juga untuk menggunakan entitas numerik.
\end{eulercomment}
\begin{eulerprompt}
>u"&#196;hnliches"
\end{eulerprompt}
\begin{euleroutput}
  Ähnliches
\end{euleroutput}
\eulersubheading{Nilai Boolean}
\begin{eulercomment}
Nilai Boolean direpresentasikan dengan 1 = benar atau 0 = salah dalam
Euler. String dapat dibandingkan, seperti halnya angka.
\end{eulercomment}
\begin{eulerprompt}
>2<1, "apel"<"banana"
\end{eulerprompt}
\begin{euleroutput}
  0
  1
\end{euleroutput}
\begin{eulercomment}
"dan" adalah operator "\&\&" dan "atau" adalah operator "\textbar{}\textbar{}", seperti
dalam bahasa C. (Kata "dan" dan "atau" hanya dapat digunakan dalam
kondisi "jika").
\end{eulercomment}
\begin{eulerprompt}
>2<E && E<3
\end{eulerprompt}
\begin{euleroutput}
  1
\end{euleroutput}
\begin{eulercomment}
Operator Boolean mematuhi aturan bahasa matriks.
\end{eulercomment}
\begin{eulerprompt}
>(1:10)>5, nonzeros(%)
\end{eulerprompt}
\begin{euleroutput}
  [0,  0,  0,  0,  0,  1,  1,  1,  1,  1]
  [6,  7,  8,  9,  10]
\end{euleroutput}
\begin{eulercomment}
Anda dapat menggunakan fungsi nonzeros() untuk mengekstrak elemen
tertentu dari sebuah vektor. Pada contoh, kita menggunakan kondisional
isprime(n).
\end{eulercomment}
\begin{eulerprompt}
>N=2|3:2:99 // N berisi elemen 2 dan bilangan2 ganjil dari 3 s.d. 99
\end{eulerprompt}
\begin{euleroutput}
  [2,  3,  5,  7,  9,  11,  13,  15,  17,  19,  21,  23,  25,  27,  29,
  31,  33,  35,  37,  39,  41,  43,  45,  47,  49,  51,  53,  55,  57,
  59,  61,  63,  65,  67,  69,  71,  73,  75,  77,  79,  81,  83,  85,
  87,  89,  91,  93,  95,  97,  99]
\end{euleroutput}
\begin{eulerprompt}
>N[nonzeros(isprime(N))] //pilih anggota2 N yang prima
\end{eulerprompt}
\begin{euleroutput}
  [2,  3,  5,  7,  11,  13,  17,  19,  23,  29,  31,  37,  41,  43,  47,
  53,  59,  61,  67,  71,  73,  79,  83,  89,  97]
\end{euleroutput}
\eulersubheading{Format Output}
\begin{eulercomment}
Format output default EMT mencetak 12 digit. Untuk memastikan bahwa
kita melihat format default, kita atur ulang formatnya.
\end{eulercomment}
\begin{eulerprompt}
>defformat; pi
\end{eulerprompt}
\begin{euleroutput}
  3.14159265359
\end{euleroutput}
\begin{eulercomment}
Secara internal, EMT menggunakan standar IEEE untuk angka ganda dengan
sekitar 16 digit desimal. Untuk melihat jumlah digit penuh, gunakan
perintah "longestformat", atau kami menggunakan operator "longest"
untuk menampilkan hasil dalam format terpanjang.
\end{eulercomment}
\begin{eulerprompt}
>longest pi
\end{eulerprompt}
\begin{euleroutput}
        3.141592653589793 
\end{euleroutput}
\begin{eulercomment}
Berikut ini adalah representasi heksadesimal internal dari angka
ganda.
\end{eulercomment}
\begin{eulerprompt}
>printhex(pi)
\end{eulerprompt}
\begin{euleroutput}
  3.243F6A8885A30*16^0
\end{euleroutput}
\begin{eulercomment}
Format output dapat diubah secara permanen dengan perintah format.
\end{eulercomment}
\begin{eulerprompt}
>format(12,5); 1/3, pi, sin(1)
\end{eulerprompt}
\begin{euleroutput}
      0.33333 
      3.14159 
      0.84147 
\end{euleroutput}
\begin{eulercomment}
Standarnya adalah format(12).
\end{eulercomment}
\begin{eulerprompt}
>format(12); 1/3
\end{eulerprompt}
\begin{euleroutput}
  0.333333333333
\end{euleroutput}
\begin{eulercomment}
Fungsi seperti "shortestformat", "shortformat", "longformat" bekerja
untuk vektor dengan cara berikut.
\end{eulercomment}
\begin{eulerprompt}
>shortestformat; random(3,8)
\end{eulerprompt}
\begin{euleroutput}
    0.66    0.2   0.89   0.28   0.53   0.31   0.44    0.3 
    0.28   0.88   0.27    0.7   0.22   0.45   0.31   0.91 
    0.19   0.46  0.095    0.6   0.43   0.73   0.47   0.32 
\end{euleroutput}
\begin{eulercomment}
Format default untuk skalar adalah format(12). Tetapi ini dapat
diubah.
\end{eulercomment}
\begin{eulerprompt}
>setscalarformat(5); pi
\end{eulerprompt}
\begin{euleroutput}
  3.1416
\end{euleroutput}
\begin{eulercomment}
Fungsi "longestformat" juga menetapkan format skalar.
\end{eulercomment}
\begin{eulerprompt}
>longestformat; pi
\end{eulerprompt}
\begin{euleroutput}
  3.141592653589793
\end{euleroutput}
\begin{eulercomment}
Sebagai referensi, berikut ini adalah daftar format output yang paling
penting.

\end{eulercomment}
\begin{eulerttcomment}
 shortestformat shortformat longformat, longestformat
 format(length,digits) goodformat(length)
 fracformat(length)
 defformat
\end{eulerttcomment}
\begin{eulercomment}

Akurasi internal EMT adalah sekitar 16 tempat desimal, yang merupakan
standar IEEE. Angka disimpan dalam format internal ini.

Tetapi format keluaran EMT dapat diatur dengan cara yang fleksibel.
\end{eulercomment}
\begin{eulerprompt}
>longestformat; pi,
\end{eulerprompt}
\begin{euleroutput}
  3.141592653589793
\end{euleroutput}
\begin{eulerprompt}
>format(10,5); pi
\end{eulerprompt}
\begin{euleroutput}
    3.14159 
\end{euleroutput}
\begin{eulercomment}
Standarnya adalah defformat().
\end{eulercomment}
\begin{eulerprompt}
>defformat; // default
\end{eulerprompt}
\begin{eulercomment}
Ada operator pendek yang hanya mencetak satu nilai. Operator
"longformat" akan mencetak semua digit angka yang valid.
\end{eulercomment}
\begin{eulerprompt}
>longest pi^2/2
\end{eulerprompt}
\begin{euleroutput}
        4.934802200544679 
\end{euleroutput}
\begin{eulercomment}
Ada juga operator singkat untuk mencetak hasil dalam format pecahan.
Kami sudah menggunakannya di atas.
\end{eulercomment}
\begin{eulerprompt}
>fraction 1+1/2+1/3+1/4
\end{eulerprompt}
\begin{euleroutput}
  25/12
\end{euleroutput}
\begin{eulercomment}
Karena format internal menggunakan cara biner untuk menyimpan angka,
maka nilai 0,1 tidak akan terwakili dengan tepat. Kesalahan bertambah
sedikit, seperti yang Anda lihat dalam perhitungan berikut ini.
\end{eulercomment}
\begin{eulerprompt}
>longest 0.1+0.1+0.1+0.1+0.1+0.1+0.1+0.1+0.1+0.1-1
\end{eulerprompt}
\begin{euleroutput}
   -1.110223024625157e-16 
\end{euleroutput}
\begin{eulercomment}
Tetapi, dengan "longformat" default, Anda tidak akan melihat hal ini.
Untuk kenyamanan, output angka yang sangat kecil adalah 0.
\end{eulercomment}
\begin{eulerprompt}
>0.1+0.1+0.1+0.1+0.1+0.1+0.1+0.1+0.1+0.1-1
\end{eulerprompt}
\begin{euleroutput}
  0
\end{euleroutput}
\eulerheading{Ekspresi}
\begin{eulercomment}
String atau nama dapat digunakan untuk menyimpan ekspresi matematika,
yang dapat dievaluasi oleh EMT. Untuk ini, gunakan tanda kurung
setelah ekspresi. Jika Anda bermaksud menggunakan string sebagai
ekspresi, gunakan konvensi untuk menamainya "fx" atau "fxy", dll.
Ekspresi lebih diutamakan daripada fungsi.

Variabel global dapat digunakan dalam evaluasi.
\end{eulercomment}
\begin{eulerprompt}
>r:=2; fx:="pi*r^2"; longest fx()
\end{eulerprompt}
\begin{euleroutput}
        12.56637061435917 
\end{euleroutput}
\begin{eulercomment}
Parameter ditetapkan ke x, y, dan z dalam urutan tersebut. Parameter
tambahan dapat ditambahkan dengan menggunakan parameter yang
ditetapkan.
\end{eulercomment}
\begin{eulerprompt}
>fx:="a*sin(x)^2"; fx(5,a=-1)
\end{eulerprompt}
\begin{euleroutput}
  -0.919535764538
\end{euleroutput}
\begin{eulercomment}
Perhatikan bahwa ekspresi akan selalu menggunakan variabel global,
meskipun ada variabel dalam fungsi dengan nama yang sama. (Jika tidak,
evaluasi ekspresi dalam fungsi dapat memberikan hasil yang sangat
membingungkan bagi pengguna yang memanggil fungsi tersebut).
\end{eulercomment}
\begin{eulerprompt}
>at:=4; function f(expr,x,at) := expr(x); ...
>f("at*x^2",3,5) // computes 4*3^2 not 5*3^2
\end{eulerprompt}
\begin{euleroutput}
  36
\end{euleroutput}
\begin{eulercomment}
Jika Anda ingin menggunakan nilai lain untuk "at" selain nilai global,
Anda perlu menambahkan "at=value".
\end{eulercomment}
\begin{eulerprompt}
>at:=4; function f(expr,x,a) := expr(x,at=a); ...
>f("at*x^2",3,5)
\end{eulerprompt}
\begin{euleroutput}
  45
\end{euleroutput}
\begin{eulercomment}
Sebagai referensi, kami menyatakan bahwa koleksi panggilan (dibahas di
tempat lain) dapat berisi ekspresi. Jadi kita dapat membuat contoh di
atas sebagai berikut.
\end{eulercomment}
\begin{eulerprompt}
>at:=4; function f(expr,x) := expr(x); ...
>f(\{\{"at*x^2",at=5\}\},3)
\end{eulerprompt}
\begin{euleroutput}
  45
\end{euleroutput}
\begin{eulercomment}
Ekspresi dalam x sering digunakan seperti halnya fungsi.\\
Perhatikan bahwa mendefinisikan fungsi dengan nama yang sama seperti
ekspresi simbolik global akan menghapus variabel ini untuk menghindari
kebingungan antara ekspresi simbolik dan fungsi.
\end{eulercomment}
\begin{eulerprompt}
>f &= 5*x;
>function f(x) := 6*x;
>f(2)
\end{eulerprompt}
\begin{euleroutput}
  12
\end{euleroutput}
\begin{eulercomment}
Sesuai dengan konvensi, ekspresi simbolik atau numerik harus diberi
nama fx, fxy, dll. Skema penamaan ini tidak boleh digunakan untuk
fungsi.
\end{eulercomment}
\begin{eulerprompt}
>fx &= diff(x^x,x); $&fx
\end{eulerprompt}
\begin{eulercomment}
Bentuk khusus dari sebuah ekspresi memungkinkan variabel apa pun
sebagai parameter tanpa nama untuk evaluasi ekspresi, bukan hanya "x",
"y", dll. Untuk ini, mulailah ekspresi dengan "@(variabel)...".
\end{eulercomment}
\begin{eulerprompt}
>"@(a,b) a^2+b^2", %(4,5)
\end{eulerprompt}
\begin{euleroutput}
  @(a,b) a^2+b^2
  41
\end{euleroutput}
\begin{eulercomment}
Hal ini memungkinkan untuk memanipulasi ekspresi dalam variabel lain
untuk fungsi EMT yang membutuhkan ekspresi dalam "x".

Cara paling dasar untuk mendefinisikan fungsi sederhana adalah dengan
menyimpan rumusnya dalam ekspresi simbolik atau numerik. Jika variabel
utamanya adalah x, ekspresi tersebut dapat dievaluasi seperti halnya
sebuah fungsi.

Seperti yang Anda lihat pada contoh berikut, variabel global terlihat
selama evaluasi.
\end{eulercomment}
\begin{eulerprompt}
>fx &= x^3-a*x;  ...
>a=1.2; fx(0.5)
\end{eulerprompt}
\begin{euleroutput}
  -0.475
\end{euleroutput}
\begin{eulercomment}
Semua variabel lain dalam ekspresi dapat ditentukan dalam evaluasi
menggunakan parameter yang ditetapkan.
\end{eulercomment}
\begin{eulerprompt}
>fx(0.5,a=1.1)
\end{eulerprompt}
\begin{euleroutput}
  -0.425
\end{euleroutput}
\begin{eulercomment}
Sebuah ekspresi tidak perlu berbentuk simbolik. Hal ini diperlukan,
jika ekspresi mengandung fungsi-fungsi, yang hanya dikenal di kernel
numerik, bukan di Maxima.

\begin{eulercomment}
\eulerheading{Matematika Simbolik}
\begin{eulercomment}
EMT melakukan matematika simbolik dengan bantuan Maxima. Untuk
detailnya, mulailah dengan tutorial berikut ini, atau telusuri
referensi untuk Maxima. Para ahli dalam Maxima harus memperhatikan
bahwa ada perbedaan dalam sintaks antara sintaks asli Maxima dan
sintaks default dari ekspresi simbolik dalam EMT.

Matematika simbolik diintegrasikan secara mulus ke dalam Euler dengan
\&. Ekspresi apapun yang dimulai dengan \& adalah sebuah ekspresi
simbolik. Ekspresi ini dievaluasi dan dicetak oleh Maxima.

Pertama-tama, Maxima memiliki aritmatika "tak terbatas" yang dapat
menangani angka yang sangat besar.
\end{eulercomment}
\begin{eulerprompt}
>$&44!
\end{eulerprompt}
\begin{eulercomment}
Dengan cara ini, Anda dapat menghitung hasil yang besar secara tepat.
Mari kita hitung

lateks: C(44,10) = \textbackslash{}frac\{44!\}\{34! \textbackslash{}cdot 10!\}
\end{eulercomment}
\begin{eulerprompt}
>$& 44!/(34!*10!) // nilai C(44,10)
\end{eulerprompt}
\begin{eulercomment}
Tentu saja, Maxima memiliki fungsi yang lebih efisien untuk hal ini
(seperti halnya bagian numerik EMT).
\end{eulercomment}
\begin{eulerprompt}
>$binomial(44,10) //menghitung C(44,10) menggunakan fungsi binomial()
\end{eulerprompt}
\begin{eulercomment}
Untuk mempelajari lebih lanjut tentang fungsi tertentu, klik dua kali
pada fungsi tersebut. Sebagai contoh, coba klik dua kali pada
"\&binomial" di baris perintah sebelumnya. Ini akan membuka dokumentasi
Maxima yang disediakan oleh pembuat program tersebut.

Anda akan mengetahui bahwa perintah-perintah berikut ini juga dapat
digunakan.

lateks: C(x,3)=\textbackslash{}frac\{x!\}\{(x-3)!3!\}=\textbackslash{}frac\{(x-2)(x-1)x\}\{6\}
\end{eulercomment}
\begin{eulerprompt}
>$binomial(x,3) // C(x,3)
\end{eulerprompt}
\begin{eulercomment}
Jika Anda ingin mengganti x dengan nilai tertentu, gunakan "with".
\end{eulercomment}
\begin{eulerprompt}
>$&binomial(x,3) with x=10 // substitusi x=10 ke C(x,3)
\end{eulerprompt}
\begin{eulercomment}
Dengan begitu, Anda dapat menggunakan solusi dari sebuah persamaan
dalam persamaan lain.

Ekspresi simbolik dicetak oleh Maxima dalam bentuk 2D. Alasannya
adalah sebuah bendera simbolik khusus dalam string.

Seperti yang telah Anda lihat pada contoh sebelumnya dan contoh
berikut, jika Anda telah menginstal LaTeX, Anda dapat mencetak
ekspresi simbolik dengan Latex. Jika tidak, perintah berikut ini akan
mengeluarkan pesan kesalahan.

Untuk mencetak ekspresi simbolik dengan LaTeX, gunakan \textdollar{} di depan \&
(atau Anda dapat menghilangkan \&) sebelum perintah. Jangan jalankan
perintah Maxima dengan \textdollar{}, jika Anda tidak memiliki LaTeX.
\end{eulercomment}
\begin{eulerprompt}
>$(3+x)/(x^2+1)
\end{eulerprompt}
\begin{eulercomment}
Ekspresi simbolik diuraikan oleh Euler. Jika Anda membutuhkan sintaks
yang kompleks dalam satu ekspresi, Anda dapat mengapit ekspresi dalam
"...". Menggunakan lebih dari satu ekspresi sederhana dimungkinkan,
tetapi sangat tidak disarankan.
\end{eulercomment}
\begin{eulerprompt}
>&"v := 5; v^2"
\end{eulerprompt}
\begin{euleroutput}
  
                                    25
  
\end{euleroutput}
\begin{eulercomment}
Untuk kelengkapan, kami menyatakan bahwa ekspresi simbolik dapat
digunakan dalam program, tetapi harus diapit dengan tanda kutip.
Selain itu, akan jauh lebih efektif untuk memanggil Maxima pada saat
kompilasi jika memungkinkan.
\end{eulercomment}
\begin{eulerprompt}
>$&expand((1+x)^4), $&factor(diff(%,x)) // diff: turunan, factor: faktor
\end{eulerprompt}
\begin{eulercomment}
Sekali lagi, \% mengacu pada hasil sebelumnya.

Untuk mempermudah, kita menyimpan solusi ke dalam sebuah variabel
simbolik. Variabel simbolik didefinisikan dengan "\&=".
\end{eulercomment}
\begin{eulerprompt}
>fx &= (x+1)/(x^4+1); $&fx
\end{eulerprompt}
\begin{eulercomment}
Ekspresi simbolik dapat digunakan dalam ekspresi simbolik lainnya.
\end{eulercomment}
\begin{eulerprompt}
>$&factor(diff(fx,x))
\end{eulerprompt}
\begin{eulercomment}
Masukan langsung dari perintah Maxima juga tersedia. Mulai baris
perintah dengan "::". Sintaks Maxima disesuaikan dengan sintaks EMT
(disebut "mode kompatibilitas").
\end{eulercomment}
\begin{eulerprompt}
>&factor(20!)
\end{eulerprompt}
\begin{euleroutput}
  
                           2432902008176640000
  
\end{euleroutput}
\begin{eulerprompt}
>::: factor(10!)
\end{eulerprompt}
\begin{euleroutput}
  
                                 8  4  2
                                2  3  5  7
  
\end{euleroutput}
\begin{eulerprompt}
>:: factor(20!)
\end{eulerprompt}
\begin{euleroutput}
  
                          18  8  4  2
                         2   3  5  7  11 13 17 19
  
\end{euleroutput}
\begin{eulercomment}
Jika Anda adalah seorang ahli dalam Maxima, Anda mungkin ingin
menggunakan sintaks asli Maxima. Anda dapat melakukan ini dengan
":::".
\end{eulercomment}
\begin{eulerprompt}
>::: av:g$ av^2;
\end{eulerprompt}
\begin{euleroutput}
  
                                     2
                                    g
  
\end{euleroutput}
\begin{eulerprompt}
>fx &= x^3*exp(x), $fx
\end{eulerprompt}
\begin{euleroutput}
  
                                   3  x
                                  x  E
  
\end{euleroutput}
\begin{eulercomment}
Variabel tersebut dapat digunakan dalam ekspresi simbolik lainnya.
Perhatikan, bahwa pada perintah berikut ini, sisi kanan dari \&=
dievaluasi sebelum penugasan ke Fx.
\end{eulercomment}
\begin{eulerprompt}
>&(fx with x=5), $%, &float(%)
\end{eulerprompt}
\begin{euleroutput}
  
                                       5
                                  125 E
  
  
                            18551.64488782208
  
\end{euleroutput}
\begin{eulerprompt}
>fx(5)
\end{eulerprompt}
\begin{euleroutput}
  18551.6448878
\end{euleroutput}
\begin{eulercomment}
Untuk mengevaluasi ekspresi dengan nilai variabel tertentu, Anda dapat
menggunakan operator "with".

Baris perintah berikut ini juga mendemonstrasikan bahwa Maxima dapat
mengevaluasi sebuah ekspresi secara numerik dengan float().
\end{eulercomment}
\begin{eulerprompt}
>&(fx with x=10)-(fx with x=5), &float(%)
\end{eulerprompt}
\begin{euleroutput}
  
                                  10        5
                            1000 E   - 125 E
  
  
                           2.20079141499189e+7
  
\end{euleroutput}
\begin{eulerprompt}
>$factor(diff(fx,x,2))
\end{eulerprompt}
\begin{eulercomment}
Untuk mendapatkan kode Latex untuk sebuah ekspresi, Anda dapat
menggunakan perintah tex.
\end{eulercomment}
\begin{eulerprompt}
>tex(fx)
\end{eulerprompt}
\begin{euleroutput}
  x^3\(\backslash\),e^\{x\}
\end{euleroutput}
\begin{eulercomment}
Ekspresi simbolik dapat dievaluasi seperti halnya ekspresi numerik.
\end{eulercomment}
\begin{eulerprompt}
>fx(0.5)
\end{eulerprompt}
\begin{euleroutput}
  0.206090158838
\end{euleroutput}
\begin{eulercomment}
Dalam ekspresi simbolik, hal ini tidak dapat dilakukan, karena Maxima
tidak mendukungnya. Sebagai gantinya, gunakan sintaks "with" (bentuk
yang lebih baik dari perintah at(...) pada Maxima).
\end{eulercomment}
\begin{eulerprompt}
>$&fx with x=1/2
\end{eulerprompt}
\begin{eulercomment}
Penugasan ini juga bisa bersifat simbolis.
\end{eulercomment}
\begin{eulerprompt}
>$&fx with x=1+t
\end{eulerprompt}
\begin{eulercomment}
Bandingkan dengan perintah "solve" numerik di Euler, yang membutuhkan
nilai awal, dan secara opsional nilai target.
\end{eulercomment}
\begin{eulerprompt}
>$&solve(x^2+x=4,x)
\end{eulerprompt}
\begin{eulercomment}
Bandingkan dengan perintah "solve" numerik di Euler, yang membutuhkan
nilai awal, dan secara opsional nilai target.
\end{eulercomment}
\begin{eulerprompt}
>solve("x^2+x",1,y=4)
\end{eulerprompt}
\begin{euleroutput}
  1.56155281281
\end{euleroutput}
\begin{eulercomment}
Nilai numerik dari solusi simbolik dapat dihitung dengan evaluasi
hasil simbolik. Euler akan membaca penugasan x= dst. Jika Anda tidak
membutuhkan hasil numerik untuk perhitungan lebih lanjut, Anda juga
bisa membiarkan Maxima menemukan nilai numeriknya.
\end{eulercomment}
\begin{eulerprompt}
>sol &= solve(x^2+2*x=4,x); $&sol, sol(), $&float(sol)
\end{eulerprompt}
\begin{euleroutput}
  [-3.23607,  1.23607]
\end{euleroutput}
\begin{eulercomment}
Untuk mendapatkan solusi simbolik yang spesifik, seseorang dapat
menggunakan "dengan" dan indeks.
\end{eulercomment}
\begin{eulerprompt}
>$&solve(x^2+x=1,x), x2 &= x with %[2]; $&x2
\end{eulerprompt}
\begin{eulercomment}
Untuk menyelesaikan sistem persamaan, gunakan vektor persamaan.
Hasilnya adalah vektor solusi.
\end{eulercomment}
\begin{eulerprompt}
>sol &= solve([x+y=3,x^2+y^2=5],[x,y]); $&sol, $&x*y with sol[1]
\end{eulerprompt}
\begin{eulercomment}
Ekspresi simbolik dapat memiliki bendera, yang menunjukkan perlakuan
khusus di Maxima. Beberapa flag dapat digunakan sebagai perintah juga,
namun ada juga yang tidak. Bendera ditambahkan dengan "\textbar{}" (bentuk yang
lebih baik dari "ev(...,flags)")
\end{eulercomment}
\begin{eulerprompt}
>$& diff((x^3-1)/(x+1),x) //turunan bentuk pecahan
>$& diff((x^3-1)/(x+1),x) | ratsimp //menyederhanakan pecahan
>$&factor(%)
\end{eulerprompt}
\begin{eulercomment}
== Sisipan Contoh Soal ==

1) Tentukan faktor dari\\
\end{eulercomment}
\begin{eulerformula}
\[
(x^2-5x+6)^2
\]
\end{eulerformula}
\begin{eulerprompt}
>$&solve((x^2-5*x+6)^2,x)
\end{eulerprompt}
\begin{eulerformula}
\[
\left[ x=3 , x=2 \right] 
\]
\end{eulerformula}
\begin{eulercomment}
2) Tentukan faktor dari\\
\end{eulercomment}
\begin{eulerformula}
\[
x^2-4
\]
\end{eulerformula}
\begin{eulerprompt}
>$&solve(x^2-4,x)
\end{eulerprompt}
\begin{eulerformula}
\[
\left[ x=-2 , x=2 \right] 
\]
\end{eulerformula}
\begin{eulercomment}
3) Tentukan hasil dari\\
\end{eulercomment}
\begin{eulerformula}
\[
C(15,2) = \frac{15!}{13! \cdot 2!}
\]
\end{eulerformula}
\begin{eulerprompt}
>$binomial(15,2) // hasil c(15,2) dengan menggunakan binomial
\end{eulerprompt}
\begin{eulerformula}
\[
105
\]
\end{eulerformula}
\begin{eulercomment}
4) Tentukan hasil dari\\
\end{eulercomment}
\begin{eulerformula}
\[
C(15,5) = \frac{15!}{10! \cdot 5!}
\]
\end{eulerformula}
\begin{eulerprompt}
>$binomial(15,5) // hasil c(15,5) dengan menggunakan binomial
\end{eulerprompt}
\begin{eulerformula}
\[
3003
\]
\end{eulerformula}
\begin{eulercomment}
5) Tentukan hasil dari\\
\end{eulercomment}
\begin{eulerformula}
\[
C(5,1) = \frac{5!}{4! \cdot 1!}
\]
\end{eulerformula}
\begin{eulerprompt}
>$binomial(5,1) // hasil c(5,1) dengan menggunakan binomial
\end{eulerprompt}
\begin{eulerformula}
\[
5
\]
\end{eulerformula}
\eulerheading{Fungsi}
\begin{eulercomment}
Dalam EMT, fungsi adalah program yang ditentukan dengan perintah
"function". Fungsi dapat berupa fungsi satu baris atau fungsi
multibaris.\\
Fungsi satu baris dapat berupa numerik atau simbolik. Fungsi satu
baris numerik didefinisikan dengan ":=".
\end{eulercomment}
\begin{eulerprompt}
>function f(x) := x*sqrt(x^2+1)
\end{eulerprompt}
\begin{eulercomment}
Sebagai gambaran umum, kami menunjukkan semua definisi yang mungkin
untuk fungsi satu baris. Sebuah fungsi dapat dievaluasi seperti halnya
fungsi Euler bawaan.
\end{eulercomment}
\begin{eulerprompt}
>f(2)
\end{eulerprompt}
\begin{euleroutput}
  4.472135955
\end{euleroutput}
\begin{eulercomment}
Fungsi ini juga dapat digunakan untuk vektor, mengikuti bahasa matriks
Euler, karena ekspresi yang digunakan dalam fungsi ini adalah vektor.
\end{eulercomment}
\begin{eulerprompt}
>f(0:0.1:1)
\end{eulerprompt}
\begin{euleroutput}
  [0,  0.100499,  0.203961,  0.313209,  0.430813,  0.559017,  0.699714,
  0.854459,  1.0245,  1.21083,  1.41421]
\end{euleroutput}
\begin{eulercomment}
Fungsi dapat diplot. Alih-alih ekspresi, kita hanya perlu memberikan
nama fungsi.

Berbeda dengan ekspresi simbolik atau numerik, nama fungsi harus
disediakan dalam bentuk string.
\end{eulercomment}
\begin{eulerprompt}
>solve("f",1,y=1)
\end{eulerprompt}
\begin{euleroutput}
  0.786151377757
\end{euleroutput}
\begin{eulercomment}
Secara default, jika Anda perlu menimpa fungsi built-in, Anda harus
menambahkan kata kunci "overwrite". Menimpa fungsi bawaan berbahaya
dan dapat menyebabkan masalah bagi fungsi lain yang bergantung pada
fungsi tersebut.

Anda masih dapat memanggil fungsi bawaan sebagai "\_...", jika fungsi
tersebut merupakan fungsi dalam inti Euler.
\end{eulercomment}
\begin{eulerprompt}
>function overwrite sin (x) := _sin(x°) // redine sine in degrees
>sin(45)
\end{eulerprompt}
\begin{euleroutput}
  0.707106781187
\end{euleroutput}
\begin{eulercomment}
Sebaiknya kita hilangkan definisi ulang sin.
\end{eulercomment}
\begin{eulerprompt}
>forget sin; sin(pi/4)
\end{eulerprompt}
\begin{euleroutput}
  0.707106781187
\end{euleroutput}
\begin{eulercomment}
== Sisipan Soal dari PDF ==

1)Find f(-3), f(-2), f(1) for this function

\end{eulercomment}
\begin{eulerformula}
\[
f(x)= x^3+7x^2-12x-3
\]
\end{eulerformula}
\begin{eulerprompt}
>function f(x):= x^3+7*x^2-12*x-3
>f(-3), f(-2), f(1)
\end{eulerprompt}
\begin{euleroutput}
  69
  41
  -7
\end{euleroutput}
\begin{eulercomment}
Jadi hasil dari f(3)= 69, f(-2)= 41, f(1)= -7

Untuk soal no 2-4, diketahui\\
\end{eulercomment}
\begin{eulerformula}
\[
f(x)=3x+1,\quad g(x)=x^2-2x-6,\quad h(x)=x^3
\]
\end{eulerformula}
\begin{eulerprompt}
>function f(x):=3*x+1
>function g(x):=x^2-2*x-6
>function h(x):=x^3
\end{eulerprompt}
\begin{eulercomment}
2)Tentuukan hasil dari\\
\end{eulercomment}
\begin{eulerformula}
\[
(f\circ g)(-1)
\]
\end{eulerformula}
\begin{eulerprompt}
>g(-1); f(%)
\end{eulerprompt}
\begin{euleroutput}
  -8
\end{euleroutput}
\begin{eulercomment}
Jadi hasilnya adalah -8

3)\\
\end{eulercomment}
\begin{eulerformula}
\[
(g\circ g)(-2)
\]
\end{eulerformula}
\begin{eulerprompt}
>g(-2); g(%)
\end{eulerprompt}
\begin{euleroutput}
  -6
\end{euleroutput}
\begin{eulercomment}
Jadi hasilnya adalah -6

4)\\
\end{eulercomment}
\begin{eulerformula}
\[
(g\circ h)(\frac{1}{2})
\]
\end{eulerformula}
\begin{eulerprompt}
>h(1/2); fraction g(%)
\end{eulerprompt}
\begin{euleroutput}
  -399/64
\end{euleroutput}
\begin{eulercomment}
Jadi hasilnya adalah -399/64

5)\\
\end{eulercomment}
\begin{eulerformula}
\[
(f\circ f)(1)
\]
\end{eulerformula}
\begin{eulerprompt}
>f(1); f(%)
\end{eulerprompt}
\begin{euleroutput}
  13
\end{euleroutput}
\begin{eulercomment}
Jadi hasilnya adalah 13

\end{eulercomment}
\eulersubheading{Parameter Default}
\begin{eulercomment}
Fungsi numerik dapat memiliki parameter default.
\end{eulercomment}
\begin{eulerprompt}
>function f(x,a=1) := a*x^2
\end{eulerprompt}
\begin{eulercomment}
Omitting this parameter uses the default value.
\end{eulercomment}
\begin{eulerprompt}
>f(4)
\end{eulerprompt}
\begin{euleroutput}
  16
\end{euleroutput}
\begin{eulercomment}
Menetapkannya akan menimpa nilai default.
\end{eulercomment}
\begin{eulerprompt}
>f(4,5)
\end{eulerprompt}
\begin{euleroutput}
  80
\end{euleroutput}
\begin{eulercomment}
Parameter yang ditetapkan juga menimpanya. Ini digunakan oleh banyak
fungsi Euler seperti plot2d, plot3d.
\end{eulercomment}
\begin{eulerprompt}
>f(4,a=1)
\end{eulerprompt}
\begin{euleroutput}
  16
\end{euleroutput}
\begin{eulercomment}
Jika sebuah variabel bukan parameter, maka variabel tersebut harus
bersifat global. Fungsi satu baris dapat melihat variabel global.
\end{eulercomment}
\begin{eulerprompt}
>function f(x) := a*x^2
>a=6; f(2)
\end{eulerprompt}
\begin{euleroutput}
  24
\end{euleroutput}
\begin{eulercomment}
Tetapi parameter yang ditetapkan akan menggantikan nilai global.

Jika argumen tidak ada dalam daftar parameter yang telah ditetapkan
sebelumnya, argumen tersebut harus dideklarasikan dengan ":="!
\end{eulercomment}
\begin{eulerprompt}
>f(2,a:=5)
\end{eulerprompt}
\begin{euleroutput}
  20
\end{euleroutput}
\begin{eulercomment}
Fungsi simbolik didefinisikan dengan "\&=". Fungsi-fungsi ini
didefinisikan dalam Euler dan Maxima, dan dapat digunakan di kedua
bahasa tersebut. Ekspresi pendefinisian dijalankan melalui Maxima
sebelum definisi.
\end{eulercomment}
\begin{eulerprompt}
>function g(x) &= x^3-x*exp(-x); $&g(x)
\end{eulerprompt}
\begin{eulercomment}
Fungsi simbolis dapat digunakan dalam ekspresi simbolis.
\end{eulercomment}
\begin{eulerprompt}
>$&diff(g(x),x), $&% with x=4/3
\end{eulerprompt}
\begin{eulercomment}
Fungsi ini juga dapat digunakan dalam ekspresi numerik. Tentu saja,
ini hanya akan berfungsi jika EMT dapat menginterpretasikan semua yang
ada di dalam fungsi.
\end{eulercomment}
\begin{eulerprompt}
>g(5+g(1))
\end{eulerprompt}
\begin{euleroutput}
  178.635099908
\end{euleroutput}
\begin{eulercomment}
Mereka dapat digunakan untuk mendefinisikan fungsi atau ekspresi
simbolis lainnya.
\end{eulercomment}
\begin{eulerprompt}
>function G(x) &= factor(integrate(g(x),x)); $&G(c) // integrate: mengintegralkan
>solve(&g(x),0.5)
\end{eulerprompt}
\begin{euleroutput}
  0.703467422498
\end{euleroutput}
\begin{eulercomment}
Hal berikut ini juga dapat digunakan, karena Euler menggunakan
ekspresi simbolik dalam fungsi g, jika tidak menemukan variabel
simbolik g, dan jika ada fungsi simbolik g.
\end{eulercomment}
\begin{eulerprompt}
>solve(&g,0.5)
\end{eulerprompt}
\begin{euleroutput}
  0.703467422498
\end{euleroutput}
\begin{eulerprompt}
>function P(x,n) &= (2*x-1)^n; $&P(x,n)
>function Q(x,n) &= (x+2)^n; $&Q(x,n)
>$&P(x,4), $&expand(%)
>P(3,4)
\end{eulerprompt}
\begin{euleroutput}
  625
\end{euleroutput}
\begin{eulerprompt}
>$&P(x,4)+ Q(x,3), $&expand(%)
>$&P(x,4)-Q(x,3), $&expand(%), $&factor(%)
>$&P(x,4)*Q(x,3), $&expand(%), $&factor(%)
>$&P(x,4)/Q(x,1), $&expand(%), $&factor(%)
>function f(x) &= x^3-x; $&f(x)
\end{eulerprompt}
\begin{eulercomment}
Dengan \&=, fungsi ini bersifat simbolis, dan dapat digunakan dalam
ekspresi simbolis lainnya.
\end{eulercomment}
\begin{eulerprompt}
>$&integrate(f(x),x)
\end{eulerprompt}
\begin{eulercomment}
Dengan := fungsi tersebut berupa angka. Contoh yang baik adalah
integral pasti seperti

lateks: f(x) = \textbackslash{}int\_1\textasciicircum{}x t\textasciicircum{}t \textbackslash{}, dt,

yang tidak dapat dievaluasi secara simbolik.

Jika kita mendefinisikan ulang fungsi tersebut dengan kata kunci
"map", maka fungsi tersebut dapat digunakan untuk vektor x. Secara
internal, fungsi tersebut dipanggil untuk semua nilai x satu kali, dan
hasilnya disimpan dalam sebuah vektor.
\end{eulercomment}
\begin{eulerprompt}
>function map f(x) := integrate("x^x",1,x)
>f(0:0.5:2)
\end{eulerprompt}
\begin{euleroutput}
  [-0.783431,  -0.410816,  0,  0.676863,  2.05045]
\end{euleroutput}
\begin{eulercomment}
Fungsi dapat memiliki nilai default untuk parameter.
\end{eulercomment}
\begin{eulerprompt}
>function mylog (x,base=10) := ln(x)/ln(base);
\end{eulerprompt}
\begin{eulercomment}
Sekarang, fungsi ini dapat dipanggil dengan atau tanpa parameter
"base".
\end{eulercomment}
\begin{eulerprompt}
>mylog(100), mylog(2^6.7,2)
\end{eulerprompt}
\begin{euleroutput}
  2
  6.7
\end{euleroutput}
\begin{eulercomment}
Selain itu, dimungkinkan untuk menggunakan parameter yang ditetapkan.
\end{eulercomment}
\begin{eulerprompt}
>mylog(E^2,base=E)
\end{eulerprompt}
\begin{euleroutput}
  2
\end{euleroutput}
\begin{eulercomment}
Sering kali, kita ingin menggunakan fungsi untuk vektor di satu
tempat, dan untuk masing-masing elemen di tempat lain. Hal ini
dimungkinkan dengan parameter vektor.
\end{eulercomment}
\begin{eulerprompt}
>function f([a,b]) &= a^2+b^2-a*b+b; $&f(a,b), $&f(x,y)
\end{eulerprompt}
\begin{eulercomment}
Fungsi simbolik seperti itu dapat digunakan untuk variabel simbolik.

Tetapi fungsi ini juga dapat digunakan untuk vektor numerik.
\end{eulercomment}
\begin{eulerprompt}
>v=[3,4]; f(v)
\end{eulerprompt}
\begin{euleroutput}
  17
\end{euleroutput}
\begin{eulercomment}
Ada juga fungsi yang murni simbolis, yang tidak dapat digunakan secara
numerik.
\end{eulercomment}
\begin{eulerprompt}
>function lapl(expr,x,y) &&= diff(expr,x,2)+diff(expr,y,2)//turunan parsial kedua
\end{eulerprompt}
\begin{euleroutput}
  
                   diff(expr, y, 2) + diff(expr, x, 2)
  
\end{euleroutput}
\begin{eulerprompt}
>$&realpart((x+I*y)^4), $&lapl(%,x,y)
\end{eulerprompt}
\begin{eulercomment}
Tetapi tentu saja, semua itu bisa digunakan dalam ekspresi simbolis
atau dalam definisi fungsi simbolis.
\end{eulercomment}
\begin{eulerprompt}
>function f(x,y) &= factor(lapl((x+y^2)^5,x,y)); $&f(x,y)
\end{eulerprompt}
\begin{eulercomment}
Untuk meringkas

- \&= mendefinisikan fungsi simbolik,\\
- := mendefinisikan fungsi numerik,\\
- \&\&= mendefinisikan fungsi simbolik murni.
\end{eulercomment}
\begin{eulercomment}
== Sisipan COntoh Soal ==\\
Diketahui dua fungsi yaitu S(x) dan T(x)

\end{eulercomment}
\begin{eulerformula}
\[
P(x)= (4x+5)^n
\]
\end{eulerformula}
\begin{eulerformula}
\[
Q(x)= (8x+4)^n
\]
\end{eulerformula}
\begin{eulerprompt}
>function P(x,n) &= (4*x+5)^n; $&P(x,n)
\end{eulerprompt}
\begin{eulerformula}
\[
\left(4\,x+5\right)^{n}
\]
\end{eulerformula}
\begin{eulerprompt}
>function Q(x,n) &= (8*x+5)^n; $&Q(x,n)
\end{eulerprompt}
\begin{eulerformula}
\[
\left(8\,x+5\right)^{n}
\]
\end{eulerformula}
\begin{eulercomment}
1) Berapa hasil P(x) dengan n=2 ditambah Q(x) dengan n=3
\end{eulercomment}
\begin{eulerprompt}
>$&P(x,2)+ Q(x,3), $&expand(%)
\end{eulerprompt}
\begin{eulerformula}
\[
512\,x^3+976\,x^2+640\,x+150
\]
\end{eulerformula}
\eulerimg{0}{images/EMT4aljabar_Pradika Larasati_23030630003-058-large.png}
\begin{eulercomment}
2) Berapa hasil P(x) dengan n=2 ditambah Q(x) dengan n=3 dengan x=2
\end{eulercomment}
\begin{eulerprompt}
>$&P(2,2)+ Q(2,3)
\end{eulerprompt}
\begin{eulerformula}
\[
9430
\]
\end{eulerformula}
\begin{eulercomment}
3) Berapa hasil P(x) dengan n=2 dikali Q(x) dengan n=3 dan factornya
\end{eulercomment}
\begin{eulerprompt}
>$&P(x,2)* Q(x,3); $&expand(%), $&factor(%)
\end{eulerprompt}
\begin{eulerformula}
\[
\left(4\,x+5\right)^2\,\left(8\,x+5\right)^3
\]
\end{eulerformula}
\eulerimg{0}{images/EMT4aljabar_Pradika Larasati_23030630003-061-large.png}
\begin{eulercomment}
4. Berapa hasil Q(x) dengan n=4
\end{eulercomment}
\begin{eulerprompt}
>$&Q(x,4), $&expand(%)
\end{eulerprompt}
\begin{eulerformula}
\[
4096\,x^4+10240\,x^3+9600\,x^2+4000\,x+625
\]
\end{eulerformula}
\eulerimg{0}{images/EMT4aljabar_Pradika Larasati_23030630003-063-large.png}
\begin{eulercomment}
5. Tentukan integral dari P(x) dengan n=2
\end{eulercomment}
\begin{eulerprompt}
>$&integrate(P(x,2),x)
\end{eulerprompt}
\begin{eulerformula}
\[
\frac{16\,x^3}{3}+20\,x^2+25\,x
\]
\end{eulerformula}
\eulerheading{Menyelesaikan Ekspresi}
\begin{eulercomment}
Ekspresi dapat diselesaikan secara numerik dan simbolik.

Untuk menyelesaikan ekspresi sederhana dari satu variabel, kita dapat
menggunakan fungsi solve(). Fungsi ini membutuhkan nilai awal untuk
memulai pencarian. Secara internal, solve() menggunakan metode secant.
\end{eulercomment}
\begin{eulerprompt}
>solve("x^2-2",1)
\end{eulerprompt}
\begin{euleroutput}
  1.41421356237
\end{euleroutput}
\begin{eulercomment}
Hal ini juga bisa digunakan untuk ekspresi simbolis. Perhatikan fungsi
berikut ini.
\end{eulercomment}
\begin{eulerprompt}
>$&solve(x^2=2,x)
>$&solve(x^2-2,x)
>$&solve(a*x^2+b*x+c=0,x)
>$&solve([a*x+b*y=c,d*x+e*y=f],[x,y])
>px &= 4*x^8+x^7-x^4-x; $&px
\end{eulerprompt}
\begin{eulercomment}
Sekarang kita mencari titik, di mana polinomialnya adalah 2. Dalam
solve(), nilai target default y=0 dapat diubah dengan variabel yang
ditetapkan.\\
Kami menggunakan y=2 dan mengeceknya dengan mengevaluasi polinomial
pada hasil sebelumnya.
\end{eulercomment}
\begin{eulerprompt}
>solve(px,1,y=2), px(%)
\end{eulerprompt}
\begin{euleroutput}
  0.966715594851
  2
\end{euleroutput}
\begin{eulercomment}
Memecahkan sebuah ekspresi simbolik dalam bentuk simbolik
mengembalikan sebuah daftar solusi. Kami menggunakan pemecah simbolik
solve() yang disediakan oleh Maxima.
\end{eulercomment}
\begin{eulerprompt}
>sol &= solve(x^2-x-1,x); $&sol
\end{eulerprompt}
\begin{eulercomment}
Cara termudah untuk mendapatkan nilai numerik adalah dengan
mengevaluasi solusi secara numerik seperti sebuah ekspresi.
\end{eulercomment}
\begin{eulerprompt}
>longest sol()
\end{eulerprompt}
\begin{euleroutput}
      -0.6180339887498949       1.618033988749895 
\end{euleroutput}
\begin{eulercomment}
Untuk menggunakan solusi secara simbolis dalam ekspresi lain, cara
termudah adalah "dengan".
\end{eulercomment}
\begin{eulerprompt}
>$&x^2 with sol[1], $&expand(x^2-x-1 with sol[2])
\end{eulerprompt}
\begin{eulercomment}
Menyelesaikan sistem persamaan secara simbolik dapat dilakukan dengan
vektor persamaan dan pemecah simbolik solve(). Jawabannya adalah
sebuah daftar daftar persamaan.
\end{eulercomment}
\begin{eulerprompt}
>$&solve([x+y=2,x^3+2*y+x=4],[x,y])
\end{eulerprompt}
\begin{eulercomment}
Fungsi f() dapat melihat variabel global. Tetapi seringkali kita ingin
menggunakan parameter lokal.

lateks: a\textasciicircum{}x-x\textasciicircum{}a = 0.1

dengan a = 3.
\end{eulercomment}
\begin{eulerprompt}
>function f(x,a) := x^a-a^x;
\end{eulerprompt}
\begin{eulercomment}
Salah satu cara untuk mengoper parameter tambahan ke f() adalah dengan
menggunakan sebuah daftar yang berisi nama fungsi dan parameternya
(cara lainnya adalah dengan menggunakan parameter titik koma).
\end{eulercomment}
\begin{eulerprompt}
>solve(\{\{"f",3\}\},2,y=0.1)
\end{eulerprompt}
\begin{euleroutput}
  2.54116291558
\end{euleroutput}
\begin{eulercomment}
Hal ini juga dapat dilakukan dengan ekspresi. Namun, elemen daftar
bernama harus digunakan. (Lebih lanjut tentang daftar dalam tutorial
tentang sintaks EMT).
\end{eulercomment}
\begin{eulerprompt}
>solve(\{\{"x^a-a^x",a=3\}\},2,y=0.1)
\end{eulerprompt}
\begin{euleroutput}
  2.54116291558
\end{euleroutput}
\begin{eulercomment}
== Sisipan Soal dari PDF ==

1)\\
\end{eulercomment}
\begin{eulerformula}
\[
\frac{1}{4}+\frac{1}{5}=\frac{1}{t}
\]
\end{eulerformula}
\begin{eulerprompt}
>$&solve(1/4+1/5=1/t,t)
\end{eulerprompt}
\begin{eulerformula}
\[
\left[ t=\frac{20}{9} \right] 
\]
\end{eulerformula}
\begin{euleroutput}
  
\end{euleroutput}
\begin{euleroutput}
  
\end{euleroutput}
\begin{eulercomment}
2)\\
\end{eulercomment}
\begin{eulerformula}
\[
\frac{1}{3}-\frac{5}{6}=\frac{1}{x}
\]
\end{eulerformula}
\begin{eulerprompt}
>$&solve(1/3-5/6=1/x,x)
\end{eulerprompt}
\begin{eulerformula}
\[
\left[ x=-2 \right] 
\]
\end{eulerformula}
\begin{eulercomment}
3)\\
\end{eulercomment}
\begin{eulerformula}
\[
\frac{x+2}{4}-\frac{x-1}{5}=15
\]
\end{eulerformula}
\begin{eulerprompt}
>$&solve(((x+2)/4-(x-1)/5)=15,x)
\end{eulerprompt}
\begin{eulerformula}
\[
\left[ x=286 \right] 
\]
\end{eulerformula}
\begin{eulercomment}
4)\\
\end{eulercomment}
\begin{eulerformula}
\[
\frac{t+1}{3}-\frac{t-1}{2}=1
\]
\end{eulerformula}
\begin{eulerprompt}
>$&solve((t+1)/3-(t-1)/2=1,t)
\end{eulerprompt}
\begin{eulerformula}
\[
\left[ t=-1 \right] 
\]
\end{eulerformula}
\begin{eulercomment}
5)\\
\end{eulercomment}
\begin{eulerformula}
\[
\frac{1}{2}+\frac{2}{x}=\frac{1}{3}+\frac{3}{x}
\]
\end{eulerformula}
\begin{eulerprompt}
>$&solve(1/2+2/x=1/3+3/x,x)
\end{eulerprompt}
\begin{eulerformula}
\[
\left[ x=6 \right] 
\]
\end{eulerformula}
\eulerheading{Menyelesaikan Pertidaksamaan.}
\begin{eulercomment}
Untuk menyelesaikan pertidaksamaan, EMT tidak akan dapat melakukannya,
melainkan dengan bantuan Maxima, yaitu dengan cara eksak (simbolik).
Perintah Maxima yang digunakan adalah fourier\_elim(), yang harus
dipanggil dengan perintah "load(fourier\_elim)" terlebih dahulu.
\end{eulercomment}
\begin{eulerprompt}
>&load(fourier_elim)
\end{eulerprompt}
\begin{euleroutput}
  
          C:/Program Files/Euler x64/maxima/share/maxima/5.35.1/share/f\(\backslash\)
  ourier_elim/fourier_elim.lisp
  
\end{euleroutput}
\begin{eulerprompt}
>$&fourier_elim([x^2 - 1>0],[x]) // x^2-1 > 0
\end{eulerprompt}
\begin{eulerformula}
\[
{\it fourier\_\_elim}\left(\left[ x^2-1>0 \right]  , \left[ x   \right] \right)
\]
\end{eulerformula}
\begin{eulerprompt}
>$&fourier_elim([x^2 - 1<0],[x]) // x^2-1 < 0
>$&fourier_elim([x^2 - 1 # 0],[x]) // x^-1 <> 0
>$&fourier_elim([x # 6],[x])
>$&fourier_elim([x < 1, x > 1],[x]) // tidak memiliki penyelesaian
>$&fourier_elim([minf < x, x < inf],[x]) // solusinya R
>$&fourier_elim([x^3 - 1 > 0],[x])
\end{eulerprompt}
\begin{eulerformula}
\[
{\it fourier\_\_elim}\left(\left[ x^3-1>0 \right]  , \left[ x   \right] \right)
\]
\end{eulerformula}
\begin{eulerprompt}
>$&fourier_elim([cos(x) < 1/2],[x]) // ??? gagal
\end{eulerprompt}
\begin{eulerformula}
\[
{\it fourier\_\_elim}\left(\left[ \cos x<\frac{1}{2} \right]  ,   \left[ x \right] \right)
\]
\end{eulerformula}
\begin{eulerprompt}
>$&fourier_elim([y-x < 5, x - y < 7, 10 < y],[x,y]) // sistem pertidaksamaan
>$&fourier_elim([y-x < 5, x - y < 7, 10 < y],[y,x])
\end{eulerprompt}
\begin{eulerformula}
\[
{\it fourier\_\_elim}\left(\left[ y-x<5 , x-y<7 , 10<y \right]  ,   \left[ y , x \right] \right)
\]
\end{eulerformula}
\begin{eulerprompt}
>$&fourier_elim((x + y < 5) and (x - y >8),[x,y]) 
\end{eulerprompt}
\begin{eulerformula}
\[
{\it fourier\_\_elim}\left(y+x<5\land x-y>8 , \left[ x , y \right]   \right)
\]
\end{eulerformula}
\begin{eulerprompt}
>$&fourier_elim(((x + y < 5) and x < 1) or  (x - y >8),[x,y])
\end{eulerprompt}
\begin{eulerformula}
\[
{\it fourier\_\_elim}\left(y+x<5\land x<1\lor x-y>8 , \left[ x , y   \right] \right)
\]
\end{eulerformula}
\begin{eulerprompt}
>&fourier_elim([max(x,y) > 6, x # 8, abs(y-1) > 12],[x,y])
\end{eulerprompt}
\begin{euleroutput}
  
         fourier_elim([max(x, y) > 6, x # 8, mabs(y - 1) > 12], [x, y])
  
\end{euleroutput}
\begin{eulerprompt}
>$&fourier_elim([(x+6)/(x-9) <= 6],[x])
\end{eulerprompt}
\begin{eulerformula}
\[
{\it fourier\_\_elim}\left(\left[ \frac{x+6}{x-9}\leq 6 \right]  ,   \left[ x \right] \right)
\]
\end{eulerformula}
\begin{eulercomment}
==Sisipan soal dari PDF \textbar{} subtopik : menyelesaikan pertidaksamaan==

Berapakah nilai x pada masing-masing soal berikut?

1)\\
\end{eulercomment}
\begin{eulerformula}
\[
\left|2x\right| \geq 6
\]
\end{eulerformula}
\begin{eulerprompt}
>$&fourier_elim([abs(2*x)>=6],[x])
\end{eulerprompt}
\begin{eulerformula}
\[
{\it fourier\_\_elim}\left(\left[ 2\,\left| x\right| \geq 6   \right]  , \left[ x \right] \right)
\]
\end{eulerformula}
\begin{eulercomment}
nilai x yang memenuhi adalah x \textgreater{}= 3, x \textless{}= -3

2)\\
\end{eulercomment}
\begin{eulerformula}
\[
\left|x-\frac{1}{4}\right|<\frac{1}{2}
\]
\end{eulerformula}
\begin{eulerprompt}
>$&fourier_elim([abs(x-1/4)<1/2],[x])
\end{eulerprompt}
\begin{eulerformula}
\[
{\it fourier\_\_elim}\left(\left[ \left| x-\frac{1}{4}\right| <  \frac{1}{2} \right]  , \left[ x \right] \right)
\]
\end{eulerformula}
\begin{eulercomment}
nilai x yang memenuhi adalah x \textgreater{} -1/4, x \textless{} 3/4

3)\\
\end{eulercomment}
\begin{eulerformula}
\[
\left|x+8\right|<9
\]
\end{eulerformula}
\begin{eulerprompt}
>$&fourier_elim([abs(x+8)<9],[x])
\end{eulerprompt}
\begin{eulerformula}
\[
{\it fourier\_\_elim}\left(\left[ \left| x+8\right| <9 \right]  ,   \left[ x \right] \right)
\]
\end{eulerformula}
\begin{eulercomment}
nilai x yang memenuhi adalah x \textgreater{} -17, x \textless{} 1

4)\\
\end{eulercomment}
\begin{eulerformula}
\[
\left|2x+3\right|\leq9
\]
\end{eulerformula}
\begin{eulerprompt}
>$&fourier_elim([abs(2*x+3)<=9],[x])
\end{eulerprompt}
\begin{eulerformula}
\[
{\it fourier\_\_elim}\left(\left[ \left| 2\,x+3\right| \leq 9   \right]  , \left[ x \right] \right)
\]
\end{eulerformula}
\begin{eulercomment}
nilai x yang memenuhi adalah x \textgreater{}= -6, x \textless{}= 3

5)\\
\end{eulercomment}
\begin{eulerformula}
\[
\left|x-5\right|\geq 0.1
\]
\end{eulerformula}
\begin{eulerprompt}
>$&fourier_elim([abs(x-5)>= 0.1],[x])
\end{eulerprompt}
\begin{eulerformula}
\[
{\it fourier\_\_elim}\left(\left[ \left| x-5\right| \geq 0.1   \right]  , \left[ x \right] \right)
\]
\end{eulerformula}
\begin{eulercomment}
nilai x yang memenuhi adalah x \textgreater{}= 5.1, x \textless{}= 4.9
\end{eulercomment}
\eulerheading{Bahasa Matriks}
\begin{eulercomment}
Dokumentasi inti EMT berisi diskusi terperinci tentang bahasa matriks
Euler.

Vektor dan matriks dimasukkan dengan tanda kurung siku, elemen
dipisahkan dengan koma, baris dipisahkan dengan titik koma.
\end{eulercomment}
\begin{eulerprompt}
>A=[1,2;3,4]
\end{eulerprompt}
\begin{euleroutput}
              1             2 
              3             4 
\end{euleroutput}
\begin{eulercomment}
Hasil kali matriks dilambangkan dengan sebuah titik.
\end{eulercomment}
\begin{eulerprompt}
>b=[3;4]
\end{eulerprompt}
\begin{euleroutput}
              3 
              4 
\end{euleroutput}
\begin{eulerprompt}
>b' // transpose b
\end{eulerprompt}
\begin{euleroutput}
  [3,  4]
\end{euleroutput}
\begin{eulerprompt}
>inv(A) //inverse A
\end{eulerprompt}
\begin{euleroutput}
             -2             1 
            1.5          -0.5 
\end{euleroutput}
\begin{eulerprompt}
>A.b //perkalian matriks
\end{eulerprompt}
\begin{euleroutput}
             11 
             25 
\end{euleroutput}
\begin{eulerprompt}
>A.inv(A)
\end{eulerprompt}
\begin{euleroutput}
              1             0 
              0             1 
\end{euleroutput}
\begin{eulercomment}
Poin utama dari bahasa matriks adalah bahwa semua fungsi dan operator
bekerja elemen demi elemen.
\end{eulercomment}
\begin{eulerprompt}
>A.A
\end{eulerprompt}
\begin{euleroutput}
              7            10 
             15            22 
\end{euleroutput}
\begin{eulerprompt}
>A^2 //perpangkatan elemen2 A
\end{eulerprompt}
\begin{euleroutput}
              1             4 
              9            16 
\end{euleroutput}
\begin{eulerprompt}
>A.A.A
\end{eulerprompt}
\begin{euleroutput}
             37            54 
             81           118 
\end{euleroutput}
\begin{eulerprompt}
>power(A,3) //perpangkatan matriks
\end{eulerprompt}
\begin{euleroutput}
             37            54 
             81           118 
\end{euleroutput}
\begin{eulerprompt}
>A/A //pembagian elemen-elemen matriks yang seletak
\end{eulerprompt}
\begin{euleroutput}
              1             1 
              1             1 
\end{euleroutput}
\begin{eulerprompt}
>A/b //pembagian elemen2 A oleh elemen2 b kolom demi kolom (karena b vektor kolom)
\end{eulerprompt}
\begin{euleroutput}
       0.333333      0.666667 
           0.75             1 
\end{euleroutput}
\begin{eulerprompt}
>A\(\backslash\)b // hasilkali invers A dan b, A^(-1)b 
\end{eulerprompt}
\begin{euleroutput}
             -2 
            2.5 
\end{euleroutput}
\begin{eulerprompt}
>inv(A).b
\end{eulerprompt}
\begin{euleroutput}
             -2 
            2.5 
\end{euleroutput}
\begin{eulerprompt}
>A\(\backslash\)A   //A^(-1)A
\end{eulerprompt}
\begin{euleroutput}
              1             0 
              0             1 
\end{euleroutput}
\begin{eulerprompt}
>inv(A).A
\end{eulerprompt}
\begin{euleroutput}
              1             0 
              0             1 
\end{euleroutput}
\begin{eulerprompt}
>A*A //perkalin elemen-elemen matriks seletak
\end{eulerprompt}
\begin{euleroutput}
              1             4 
              9            16 
\end{euleroutput}
\begin{eulercomment}
Ini bukan hasil kali matriks, tetapi perkalian elemen demi elemen. Hal
yang sama berlaku untuk vektor.
\end{eulercomment}
\begin{eulerprompt}
>b^2 // perpangkatan elemen-elemen matriks/vektor
\end{eulerprompt}
\begin{euleroutput}
              9 
             16 
\end{euleroutput}
\begin{eulercomment}
Jika salah satu operan adalah vektor atau skalar, maka operan tersebut
akan diperluas dengan cara alami.
\end{eulercomment}
\begin{eulerprompt}
>2*A
\end{eulerprompt}
\begin{euleroutput}
              2             4 
              6             8 
\end{euleroutput}
\begin{eulercomment}
Misalnya, jika operan adalah vektor kolom, elemen-elemennya diterapkan
ke semua baris A.
\end{eulercomment}
\begin{eulerprompt}
>[1,2]*A
\end{eulerprompt}
\begin{euleroutput}
              1             4 
              3             8 
\end{euleroutput}
\begin{eulercomment}
Jika ini adalah vektor baris, vektor ini diterapkan ke semua kolom A.
\end{eulercomment}
\begin{eulerprompt}
>A*[2,3]
\end{eulerprompt}
\begin{euleroutput}
              2             6 
              6            12 
\end{euleroutput}
\begin{eulercomment}
Kita dapat membayangkan perkalian ini seolah-olah vektor baris v telah
diduplikasi untuk membentuk matriks dengan ukuran yang sama dengan A.
\end{eulercomment}
\begin{eulerprompt}
>dup([1,2],2) // dup: menduplikasi/menggandakan vektor [1,2] sebanyak 2 kali (baris)
\end{eulerprompt}
\begin{euleroutput}
              1             2 
              1             2 
\end{euleroutput}
\begin{eulerprompt}
>A*dup([1,2],2) 
\end{eulerprompt}
\begin{euleroutput}
              1             4 
              3             8 
\end{euleroutput}
\begin{eulercomment}
Hal ini juga berlaku untuk dua vektor di mana satu vektor adalah
vektor baris dan yang lainnya adalah vektor kolom. Kami menghitung i*j
untuk i, j dari 1 sampai 5. Caranya adalah dengan mengalikan 1:5
dengan transposenya. Bahasa matriks Euler secara otomatis menghasilkan
sebuah tabel nilai.
\end{eulercomment}
\begin{eulerprompt}
>(1:5)*(1:5)' // hasilkali elemen-elemen vektor baris dan vektor kolom
\end{eulerprompt}
\begin{euleroutput}
              1             2             3             4             5 
              2             4             6             8            10 
              3             6             9            12            15 
              4             8            12            16            20 
              5            10            15            20            25 
\end{euleroutput}
\begin{eulercomment}
Sekali lagi, ingatlah bahwa ini bukan produk matriks!
\end{eulercomment}
\begin{eulerprompt}
>(1:5).(1:5)' // hasilkali vektor baris dan vektor kolom
\end{eulerprompt}
\begin{euleroutput}
  55
\end{euleroutput}
\begin{eulerprompt}
>sum((1:5)*(1:5)) // sama hasilnya
\end{eulerprompt}
\begin{euleroutput}
  55
\end{euleroutput}
\begin{eulercomment}
Bahkan operator seperti \textless{} atau == bekerja dengan cara yang sama.
\end{eulercomment}
\begin{eulerprompt}
>(1:10)<6 // menguji elemen-elemen yang kurang dari 6
\end{eulerprompt}
\begin{euleroutput}
  [1,  1,  1,  1,  1,  0,  0,  0,  0,  0]
\end{euleroutput}
\begin{eulercomment}
Sebagai contoh, kita dapat menghitung jumlah elemen yang memenuhi
kondisi tertentu dengan fungsi sum().
\end{eulercomment}
\begin{eulerprompt}
>sum((1:10)<6) // banyak elemen yang kurang dari 6
\end{eulerprompt}
\begin{euleroutput}
  5
\end{euleroutput}
\begin{eulercomment}
Euler memiliki operator perbandingan, seperti "==", yang memeriksa
kesetaraan.

Kita mendapatkan vektor 0 dan 1, di mana 1 berarti benar.
\end{eulercomment}
\begin{eulerprompt}
>t=(1:10)^2; t==25 //menguji elemen2 t yang sama dengan 25 (hanya ada 1)
\end{eulerprompt}
\begin{euleroutput}
  [0,  0,  0,  0,  1,  0,  0,  0,  0,  0]
\end{euleroutput}
\begin{eulercomment}
Dari vektor seperti itu, "nonzeros" memilih elemen bukan nol.

Dalam hal ini, kita mendapatkan indeks semua elemen yang lebih besar
dari 50.
\end{eulercomment}
\begin{eulerprompt}
>nonzeros(t>50) //indeks elemen2 t yang lebih besar daripada 50
\end{eulerprompt}
\begin{euleroutput}
  [8,  9,  10]
\end{euleroutput}
\begin{eulercomment}
Tentu saja, kita dapat menggunakan vektor indeks ini untuk mendapatkan
nilai yang sesuai dalam t.
\end{eulercomment}
\begin{eulerprompt}
>t[nonzeros(t>50)] //elemen2 t yang lebih besar daripada 50
\end{eulerprompt}
\begin{euleroutput}
  [64,  81,  100]
\end{euleroutput}
\begin{eulercomment}
Sebagai contoh, mari kita cari semua kuadrat dari angka 1 sampai 1000,
yaitu 5 modulo 11 dan 3 modulo 13.
\end{eulercomment}
\begin{eulerprompt}
>t=1:1000; nonzeros(mod(t^2,11)==5 && mod(t^2,13)==3)
\end{eulerprompt}
\begin{euleroutput}
  [4,  48,  95,  139,  147,  191,  238,  282,  290,  334,  381,  425,
  433,  477,  524,  568,  576,  620,  667,  711,  719,  763,  810,  854,
  862,  906,  953,  997]
\end{euleroutput}
\begin{eulercomment}
EMT tidak sepenuhnya efektif untuk komputasi bilangan bulat. EMT
menggunakan floating point presisi ganda secara internal. Akan tetapi,
hal ini sering kali sangat berguna.

Kita dapat memeriksa bilangan prima. Mari kita cari tahu, berapa
banyak kuadrat ditambah 1 yang merupakan bilangan prima.
\end{eulercomment}
\begin{eulerprompt}
>t=1:1000; length(nonzeros(isprime(t^2+1)))
\end{eulerprompt}
\begin{euleroutput}
  112
\end{euleroutput}
\begin{eulercomment}
Fungsi nonzeros() hanya bekerja untuk vektor. Untuk matriks, ada
mnonzeros().
\end{eulercomment}
\begin{eulerprompt}
>seed(2); A=random(3,4)
\end{eulerprompt}
\begin{euleroutput}
       0.765761      0.401188      0.406347      0.267829 
        0.13673      0.390567      0.495975      0.952814 
       0.548138      0.006085      0.444255      0.539246 
\end{euleroutput}
\begin{eulercomment}
Ini mengembalikan indeks elemen, yang bukan nol.
\end{eulercomment}
\begin{eulerprompt}
>k=mnonzeros(A<0.4) //indeks elemen2 A yang kurang dari 0,4
\end{eulerprompt}
\begin{euleroutput}
              1             4 
              2             1 
              2             2 
              3             2 
\end{euleroutput}
\begin{eulercomment}
Indeks ini dapat digunakan untuk menetapkan elemen ke suatu nilai.
\end{eulercomment}
\begin{eulerprompt}
>mset(A,k,0) //mengganti elemen2 suatu matriks pada indeks tertentu
\end{eulerprompt}
\begin{euleroutput}
       0.765761      0.401188      0.406347             0 
              0             0      0.495975      0.952814 
       0.548138             0      0.444255      0.539246 
\end{euleroutput}
\begin{eulercomment}
Fungsi mset() juga dapat mengatur elemen-elemen pada indeks ke
entri-entri matriks lain.
\end{eulercomment}
\begin{eulerprompt}
>mset(A,k,-random(size(A)))
\end{eulerprompt}
\begin{euleroutput}
       0.765761      0.401188      0.406347     -0.126917 
      -0.122404     -0.691673      0.495975      0.952814 
       0.548138     -0.483902      0.444255      0.539246 
\end{euleroutput}
\begin{eulercomment}
Dan dimungkinkan untuk mendapatkan elemen-elemen dalam vektor.
\end{eulercomment}
\begin{eulerprompt}
>mget(A,k)
\end{eulerprompt}
\begin{euleroutput}
  [0.267829,  0.13673,  0.390567,  0.006085]
\end{euleroutput}
\begin{eulercomment}
Fungsi lain yang berguna adalah extrema, yang mengembalikan nilai
minimal dan maksimal di setiap baris matriks dan posisinya.
\end{eulercomment}
\begin{eulerprompt}
>ex=extrema(A)
\end{eulerprompt}
\begin{euleroutput}
       0.267829             4      0.765761             1 
        0.13673             1      0.952814             4 
       0.006085             2      0.548138             1 
\end{euleroutput}
\begin{eulercomment}
Kita bisa menggunakan ini untuk mengekstrak nilai maksimal dalam
setiap baris.
\end{eulercomment}
\begin{eulerprompt}
>ex[,3]'
\end{eulerprompt}
\begin{euleroutput}
  [0.765761,  0.952814,  0.548138]
\end{euleroutput}
\begin{eulercomment}
Ini, tentu saja, sama dengan fungsi max().
\end{eulercomment}
\begin{eulerprompt}
>max(A)'
\end{eulerprompt}
\begin{euleroutput}
  [0.765761,  0.952814,  0.548138]
\end{euleroutput}
\begin{eulercomment}
Tetapi dengan mget(), kita dapat mengekstrak indeks dan menggunakan
informasi ini untuk mengekstrak elemen-elemen pada posisi yang sama
dari matriks yang lain.
\end{eulercomment}
\begin{eulerprompt}
>j=(1:rows(A))'|ex[,4], mget(-A,j)
\end{eulerprompt}
\begin{euleroutput}
              1             1 
              2             4 
              3             1 
  [-0.765761,  -0.952814,  -0.548138]
\end{euleroutput}
\begin{eulercomment}
== Sisipan Soal dari PDF ==
\end{eulercomment}
\begin{eulerprompt}
>A=[3,-1;5,4]
\end{eulerprompt}
\begin{euleroutput}
              3            -1 
              5             4 
\end{euleroutput}
\begin{eulerprompt}
>B=[-2,6;1,-3]
\end{eulerprompt}
\begin{euleroutput}
             -2             6 
              1            -3 
\end{euleroutput}
\begin{eulercomment}
Find each of the following\\
1)A+B
\end{eulercomment}
\begin{eulerprompt}
>A+B
\end{eulerprompt}
\begin{euleroutput}
              1             5 
              6             1 
\end{euleroutput}
\begin{eulercomment}
2)B-A
\end{eulercomment}
\begin{eulerprompt}
>B-A
\end{eulerprompt}
\begin{euleroutput}
             -5             7 
             -4            -7 
\end{euleroutput}
\begin{eulercomment}
AB
\end{eulercomment}
\begin{eulerprompt}
>A.B
\end{eulerprompt}
\begin{euleroutput}
             -7            21 
             -6            18 
\end{euleroutput}
\begin{eulercomment}
3)2A+3B
\end{eulercomment}
\begin{eulerprompt}
>2*A+3*B
\end{eulerprompt}
\begin{euleroutput}
              0            16 
             13            -1 
\end{euleroutput}
\begin{eulercomment}
4)BA
\end{eulercomment}
\begin{eulerprompt}
>B.A
\end{eulerprompt}
\begin{euleroutput}
             24            26 
            -12           -13 
\end{euleroutput}
\begin{eulercomment}
5)Invers (A).B
\end{eulercomment}
\begin{eulerprompt}
>A=[4,2;3,-1]
\end{eulerprompt}
\begin{euleroutput}
              4             2 
              3            -1 
\end{euleroutput}
\begin{eulerprompt}
>B=[11,2;1,0]
\end{eulerprompt}
\begin{euleroutput}
             11             2 
              1             0 
\end{euleroutput}
\begin{eulerprompt}
>inv(A).B
\end{eulerprompt}
\begin{euleroutput}
            1.3           0.2 
            2.9           0.6 
\end{euleroutput}
\eulerheading{Fungsi Matriks Lainnya (Membangun Matriks)}
\begin{eulercomment}
Untuk membangun sebuah matriks, kita dapat menumpuk satu matriks di
atas matriks lainnya. Jika keduanya tidak memiliki jumlah kolom yang
sama, kolom yang lebih pendek akan diisi dengan 0.
\end{eulercomment}
\begin{eulerprompt}
>v=1:3; v_v
\end{eulerprompt}
\begin{euleroutput}
              1             2             3 
              1             2             3 
\end{euleroutput}
\begin{eulercomment}
Demikian juga, kita dapat melampirkan matriks ke matriks lain secara
berdampingan, jika keduanya memiliki jumlah baris yang sama.
\end{eulercomment}
\begin{eulerprompt}
>A=random(3,4); A|v'
\end{eulerprompt}
\begin{euleroutput}
       0.032444     0.0534171      0.595713      0.564454             1 
        0.83916      0.175552      0.396988       0.83514             2 
      0.0257573      0.658585      0.629832      0.770895             3 
\end{euleroutput}
\begin{eulercomment}
Jika keduanya tidak memiliki jumlah baris yang sama, matriks yang
lebih pendek diisi dengan 0.

Ada pengecualian untuk aturan ini. Bilangan real yang dilampirkan pada
sebuah matriks akan digunakan sebagai kolom yang diisi dengan bilangan
real tersebut.
\end{eulercomment}
\begin{eulerprompt}
>A|1
\end{eulerprompt}
\begin{euleroutput}
       0.032444     0.0534171      0.595713      0.564454             1 
        0.83916      0.175552      0.396988       0.83514             1 
      0.0257573      0.658585      0.629832      0.770895             1 
\end{euleroutput}
\begin{eulercomment}
Dimungkinkan untuk membuat matriks vektor baris dan kolom.
\end{eulercomment}
\begin{eulerprompt}
>[v;v]
\end{eulerprompt}
\begin{euleroutput}
              1             2             3 
              1             2             3 
\end{euleroutput}
\begin{eulerprompt}
>[v',v']
\end{eulerprompt}
\begin{euleroutput}
              1             1 
              2             2 
              3             3 
\end{euleroutput}
\begin{eulercomment}
Tujuan utama dari hal ini adalah untuk menginterpretasikan vektor
ekspresi untuk vektor kolom.
\end{eulercomment}
\begin{eulerprompt}
>"[x,x^2]"(v')
\end{eulerprompt}
\begin{euleroutput}
              1             1 
              2             4 
              3             9 
\end{euleroutput}
\begin{eulercomment}
Untuk mendapatkan ukuran A, kita dapat menggunakan fungsi berikut ini.
\end{eulercomment}
\begin{eulerprompt}
>C=zeros(2,4); rows(C), cols(C), size(C), length(C)
\end{eulerprompt}
\begin{euleroutput}
  2
  4
  [2,  4]
  4
\end{euleroutput}
\begin{eulercomment}
Untuk vektor, terdapat length().
\end{eulercomment}
\begin{eulerprompt}
>length(2:10)
\end{eulerprompt}
\begin{euleroutput}
  9
\end{euleroutput}
\begin{eulercomment}
Ada banyak fungsi lain yang menghasilkan matriks.
\end{eulercomment}
\begin{eulerprompt}
>ones(2,2)
\end{eulerprompt}
\begin{euleroutput}
              1             1 
              1             1 
\end{euleroutput}
\begin{eulercomment}
Ini juga dapat digunakan dengan satu parameter. Untuk mendapatkan
vektor dengan angka selain 1, gunakan yang berikut ini.
\end{eulercomment}
\begin{eulerprompt}
>ones(5)*6
\end{eulerprompt}
\begin{euleroutput}
  [6,  6,  6,  6,  6]
\end{euleroutput}
\begin{eulercomment}
Matriks angka acak juga dapat dibuat dengan acak (distribusi seragam)
atau normal (distribusi Gauß).
\end{eulercomment}
\begin{eulerprompt}
>random(2,2)
\end{eulerprompt}
\begin{euleroutput}
        0.66566      0.831835 
          0.977      0.544258 
\end{euleroutput}
\begin{eulercomment}
Berikut ini adalah fungsi lain yang berguna, yang merestrukturisasi
elemen-elemen matriks menjadi matriks lain.
\end{eulercomment}
\begin{eulerprompt}
>redim(1:9,3,3) // menyusun elemen2 1, 2, 3, ..., 9 ke bentuk matriks 3x3
\end{eulerprompt}
\begin{euleroutput}
              1             2             3 
              4             5             6 
              7             8             9 
\end{euleroutput}
\begin{eulercomment}
Dengan fungsi berikut, kita dapat menggunakan fungsi ini dan fungsi
dup untuk menulis fungsi rep(), yang mengulang sebuah vektor sebanyak
n kali.
\end{eulercomment}
\begin{eulerprompt}
>function rep(v,n) := redim(dup(v,n),1,n*cols(v))
\end{eulerprompt}
\begin{eulercomment}
Mari kita uji.
\end{eulercomment}
\begin{eulerprompt}
>rep(1:3,5)
\end{eulerprompt}
\begin{euleroutput}
  [1,  2,  3,  1,  2,  3,  1,  2,  3,  1,  2,  3,  1,  2,  3]
\end{euleroutput}
\begin{eulercomment}
Fungsi multdup() menduplikasi elemen-elemen sebuah vektor.
\end{eulercomment}
\begin{eulerprompt}
>multdup(1:3,5), multdup(1:3,[2,3,2])
\end{eulerprompt}
\begin{euleroutput}
  [1,  1,  1,  1,  1,  2,  2,  2,  2,  2,  3,  3,  3,  3,  3]
  [1,  1,  2,  2,  2,  3,  3]
\end{euleroutput}
\begin{eulercomment}
Fungsi flipx() dan flipy() membalik urutan baris atau kolom dari
sebuah matriks. Misalnya, fungsi flipx() membalik secara horizontal.
\end{eulercomment}
\begin{eulerprompt}
>flipx(1:5) //membalik elemen2 vektor baris
\end{eulerprompt}
\begin{euleroutput}
  [5,  4,  3,  2,  1]
\end{euleroutput}
\begin{eulercomment}
Untuk rotasi, Euler memiliki rotleft() dan rotright().
\end{eulercomment}
\begin{eulerprompt}
>rotleft(1:5) // memutar elemen2 vektor baris
\end{eulerprompt}
\begin{euleroutput}
  [2,  3,  4,  5,  1]
\end{euleroutput}
\begin{eulercomment}
Fungsi khusus adalah drop(v,i), yang menghapus elemen dengan indeks di
i dari vektor v.
\end{eulercomment}
\begin{eulerprompt}
>drop(10:20,3)
\end{eulerprompt}
\begin{euleroutput}
  [10,  11,  13,  14,  15,  16,  17,  18,  19,  20]
\end{euleroutput}
\begin{eulercomment}
Perhatikan bahwa vektor i dalam drop(v,i) merujuk pada indeks elemen
dalam v, bukan nilai elemen. Jika Anda ingin menghapus elemen, Anda
harus menemukan elemen-elemen tersebut terlebih dahulu. Fungsi
indexof(v,x) dapat digunakan untuk menemukan elemen x dalam vektor
terurut v.
\end{eulercomment}
\begin{eulerprompt}
>v=primes(50), i=indexof(v,10:20), drop(v,i)
\end{eulerprompt}
\begin{euleroutput}
  [2,  3,  5,  7,  11,  13,  17,  19,  23,  29,  31,  37,  41,  43,  47]
  [0,  5,  0,  6,  0,  0,  0,  7,  0,  8,  0]
  [2,  3,  5,  7,  23,  29,  31,  37,  41,  43,  47]
\end{euleroutput}
\begin{eulercomment}
Seperti yang Anda lihat, tidak ada salahnya menyertakan indeks di luar
jangkauan (seperti 0), indeks ganda, atau indeks yang tidak diurutkan.
\end{eulercomment}
\begin{eulerprompt}
>drop(1:10,shuffle([0,0,5,5,7,12,12]))
\end{eulerprompt}
\begin{euleroutput}
  [1,  2,  3,  4,  6,  8,  9,  10]
\end{euleroutput}
\begin{eulercomment}
Ada beberapa fungsi khusus untuk mengatur diagonal atau menghasilkan
matriks diagonal.

Kita mulai dengan matriks identitas.
\end{eulercomment}
\begin{eulerprompt}
>A=id(5) // matriks identitas 5x5
\end{eulerprompt}
\begin{euleroutput}
              1             0             0             0             0 
              0             1             0             0             0 
              0             0             1             0             0 
              0             0             0             1             0 
              0             0             0             0             1 
\end{euleroutput}
\begin{eulercomment}
Kemudian, kami menetapkan diagonal bawah (-1) ke 1:4.
\end{eulercomment}
\begin{eulerprompt}
>setdiag(A,-1,1:4) //mengganti diagonal di bawah diagonal utama
\end{eulerprompt}
\begin{euleroutput}
              1             0             0             0             0 
              1             1             0             0             0 
              0             2             1             0             0 
              0             0             3             1             0 
              0             0             0             4             1 
\end{euleroutput}
\begin{eulercomment}
Perhatikan bahwa kita tidak mengubah matriks A. Kita mendapatkan
sebuah matriks baru sebagai hasil dari setdiag().

Berikut adalah sebuah fungsi yang mengembalikan sebuah matriks
tri-diagonal.
\end{eulercomment}
\begin{eulerprompt}
>function tridiag (n,a,b,c) := setdiag(setdiag(b*id(n),1,c),-1,a); ...
>tridiag(5,1,2,3)
\end{eulerprompt}
\begin{euleroutput}
              2             3             0             0             0 
              1             2             3             0             0 
              0             1             2             3             0 
              0             0             1             2             3 
              0             0             0             1             2 
\end{euleroutput}
\begin{eulercomment}
Diagonal sebuah matriks juga dapat diekstrak dari matriks. Untuk
mendemonstrasikan hal ini, kami merestrukturisasi vektor 1:9 menjadi
matriks 3x3.
\end{eulercomment}
\begin{eulerprompt}
>A=redim(1:9,3,3)
\end{eulerprompt}
\begin{euleroutput}
              1             2             3 
              4             5             6 
              7             8             9 
\end{euleroutput}
\begin{eulercomment}
Sekarang kita bisa mengekstrak diagonal.
\end{eulercomment}
\begin{eulerprompt}
>d=getdiag(A,0)
\end{eulerprompt}
\begin{euleroutput}
  [1,  5,  9]
\end{euleroutput}
\begin{eulercomment}
Contoh: Kita dapat membagi matriks dengan diagonalnya. Bahasa matriks
memperhatikan bahwa vektor kolom d diterapkan ke matriks baris demi
baris.
\end{eulercomment}
\begin{eulerprompt}
>fraction A/d'
\end{eulerprompt}
\begin{euleroutput}
          1         2         3 
        4/5         1       6/5 
        7/9       8/9         1 
\end{euleroutput}
\eulerheading{Vektorisasi}
\begin{eulercomment}
Hampir semua fungsi di Euler juga dapat digunakan untuk input matriks
dan vektor, jika hal ini masuk akal.

Sebagai contoh, fungsi sqrt() menghitung akar kuadrat dari semua
elemen vektor atau matriks.
\end{eulercomment}
\begin{eulerprompt}
>sqrt(1:3)
\end{eulerprompt}
\begin{euleroutput}
  [1,  1.41421,  1.73205]
\end{euleroutput}
\begin{eulercomment}
Jadi, Anda dapat dengan mudah membuat tabel nilai. Ini adalah salah
satu cara untuk memplot sebuah fungsi (alternatif lainnya menggunakan
ekspresi).
\end{eulercomment}
\begin{eulerprompt}
>x=1:0.01:5; y=log(x)/x^2; // terlalu panjang untuk ditampikan
\end{eulerprompt}
\begin{eulercomment}
Dengan ini dan operator titik dua a:delta:b, vektor nilai fungsi dapat
dihasilkan dengan mudah.

Pada contoh berikut, kita membuat vektor nilai t[i] dengan jarak 0.1
dari -1 hingga 1. Kemudian kita membuat vektor nilai dari fungsi

lateks: s = t\textasciicircum{}3-t
\end{eulercomment}
\begin{eulerprompt}
>t=-1:0.1:1; s=t^3-t
\end{eulerprompt}
\begin{euleroutput}
  [0,  0.171,  0.288,  0.357,  0.384,  0.375,  0.336,  0.273,  0.192,
  0.099,  0,  -0.099,  -0.192,  -0.273,  -0.336,  -0.375,  -0.384,
  -0.357,  -0.288,  -0.171,  0]
\end{euleroutput}
\begin{eulercomment}
EMT memperluas operator untuk skalar, vektor, dan matriks dengan cara
yang jelas.

Misalnya, vektor kolom dikalikan vektor baris diperluas menjadi
matriks, jika operator diterapkan. Berikut ini, v' adalah vektor yang
ditransposisikan (vektor kolom).
\end{eulercomment}
\begin{eulerprompt}
>shortest (1:5)*(1:5)'
\end{eulerprompt}
\begin{euleroutput}
       1      2      3      4      5 
       2      4      6      8     10 
       3      6      9     12     15 
       4      8     12     16     20 
       5     10     15     20     25 
\end{euleroutput}
\begin{eulercomment}
Perhatikan, bahwa ini sangat berbeda dengan hasil kali matriks. Hasil
kali matriks dilambangkan dengan sebuah titik "." dalam EMT.
\end{eulercomment}
\begin{eulerprompt}
>(1:5).(1:5)'
\end{eulerprompt}
\begin{euleroutput}
  55
\end{euleroutput}
\begin{eulercomment}
Secara default, vektor baris dicetak dalam format ringkas.
\end{eulercomment}
\begin{eulerprompt}
>[1,2,3,4]
\end{eulerprompt}
\begin{euleroutput}
  [1,  2,  3,  4]
\end{euleroutput}
\begin{eulercomment}
Untuk matriks, operator khusus . menyatakan perkalian matriks, dan A'
menyatakan transposisi. Matriks 1x1 dapat digunakan seperti halnya
bilangan real.
\end{eulercomment}
\begin{eulerprompt}
>v:=[1,2]; v.v', %^2
\end{eulerprompt}
\begin{euleroutput}
  5
  25
\end{euleroutput}
\begin{eulercomment}
Untuk mentransposisikan matriks, kita menggunakan apostrof.
\end{eulercomment}
\begin{eulerprompt}
>v=1:4; v'
\end{eulerprompt}
\begin{euleroutput}
              1 
              2 
              3 
              4 
\end{euleroutput}
\begin{eulercomment}
Jadi kita dapat menghitung matriks A dikali vektor b.
\end{eulercomment}
\begin{eulerprompt}
>A=[1,2,3,4;5,6,7,8]; A.v'
\end{eulerprompt}
\begin{euleroutput}
             30 
             70 
\end{euleroutput}
\begin{eulercomment}
Perhatikan bahwa v masih merupakan vektor baris. Jadi v'.v berbeda
dengan v.v'.
\end{eulercomment}
\begin{eulerprompt}
>v'.v
\end{eulerprompt}
\begin{euleroutput}
              1             2             3             4 
              2             4             6             8 
              3             6             9            12 
              4             8            12            16 
\end{euleroutput}
\begin{eulercomment}
v.v' menghitung norma v kuadrat untuk vektor baris v. Hasilnya adalah
vektor 1x1, yang berfungsi seperti bilangan real.
\end{eulercomment}
\begin{eulerprompt}
>v.v'
\end{eulerprompt}
\begin{euleroutput}
  30
\end{euleroutput}
\begin{eulercomment}
Ada juga norma fungsi (bersama dengan banyak fungsi Aljabar Linier
lainnya).
\end{eulercomment}
\begin{eulerprompt}
>norm(v)^2
\end{eulerprompt}
\begin{euleroutput}
  30
\end{euleroutput}
\begin{eulercomment}
Operator dan fungsi mematuhi bahasa matriks Euler.

Berikut ini adalah ringkasan aturannya.

- Sebuah fungsi yang diterapkan pada vektor atau matriks diterapkan
pada setiap elemen.

- Operator yang beroperasi pada dua matriks dengan ukuran yang sama
diterapkan secara berpasangan pada elemen-elemen matriks.

- Jika dua matriks memiliki dimensi yang berbeda, keduanya diperluas
dengan cara yang masuk akal, sehingga memiliki ukuran yang sama.

Misalnya, nilai skalar dikalikan vektor mengalikan nilai tersebut
dengan setiap elemen vektor. Atau matriks dikali vektor (dengan *,
bukan .) memperluas vektor ke ukuran matriks dengan menduplikasinya.

Berikut ini adalah kasus sederhana dengan operator \textasciicircum{}.
\end{eulercomment}
\begin{eulerprompt}
>[1,2,3]^2
\end{eulerprompt}
\begin{euleroutput}
  [1,  4,  9]
\end{euleroutput}
\begin{eulercomment}
Ini adalah kasus yang lebih rumit. Vektor baris dikalikan vektor kolom
memperluas keduanya dengan menduplikasi.
\end{eulercomment}
\begin{eulerprompt}
>v:=[1,2,3]; v*v'
\end{eulerprompt}
\begin{euleroutput}
              1             2             3 
              2             4             6 
              3             6             9 
\end{euleroutput}
\begin{eulercomment}
Perhatikan bahwa hasil kali skalar menggunakan hasil kali matriks,
bukan tanda *!
\end{eulercomment}
\begin{eulerprompt}
>v.v'
\end{eulerprompt}
\begin{euleroutput}
  14
\end{euleroutput}
\begin{eulercomment}
Ada banyak fungsi untuk matriks. Kami memberikan daftar singkat. Anda
harus membaca dokumentasi untuk informasi lebih lanjut mengenai
perintah-perintah ini.

\end{eulercomment}
\begin{eulerttcomment}
  sum,prod menghitung jumlah dan hasil kali dari baris-baris
  cumsum,cumprod melakukan hal yang sama secara kumulatif
  menghitung nilai ekstrem dari setiap baris
  extrema mengembalikan vektor dengan informasi ekstrem
  diag(A,i) mengembalikan diagonal ke-i
  setdiag(A,i,v) menetapkan diagonal ke-i
  id(n) matriks identitas
  det(A) determinan
  charpoly(A) polinomial karakteristik
  eigenvalues(A) nilai eigen
\end{eulerttcomment}
\begin{eulerprompt}
>v*v, sum(v*v), cumsum(v*v)
\end{eulerprompt}
\begin{euleroutput}
  [1,  4,  9]
  14
  [1,  5,  14]
\end{euleroutput}
\begin{eulercomment}
Operator : menghasilkan vektor baris dengan spasi yang sama, opsional
dengan ukuran langkah.
\end{eulercomment}
\begin{eulerprompt}
>1:4, 1:2:10
\end{eulerprompt}
\begin{euleroutput}
  [1,  2,  3,  4]
  [1,  3,  5,  7,  9]
\end{euleroutput}
\begin{eulercomment}
Untuk menggabungkan matriks dan vektor, terdapat operator "\textbar{}" dan "\_".
\end{eulercomment}
\begin{eulerprompt}
>[1,2,3]|[4,5], [1,2,3]_1
\end{eulerprompt}
\begin{euleroutput}
  [1,  2,  3,  4,  5]
              1             2             3 
              1             1             1 
\end{euleroutput}
\begin{eulercomment}
Elemen-elemen dari sebuah matriks disebut dengan "A[i,j]".
\end{eulercomment}
\begin{eulerprompt}
>A:=[1,2,3;4,5,6;7,8,9]; A[2,3]
\end{eulerprompt}
\begin{euleroutput}
  6
\end{euleroutput}
\begin{eulercomment}
Untuk vektor baris atau kolom, v[i] adalah elemen ke-i dari vektor
tersebut. Untuk matriks, ini mengembalikan baris ke-i dari matriks.
\end{eulercomment}
\begin{eulerprompt}
>v:=[2,4,6,8]; v[3], A[3]
\end{eulerprompt}
\begin{euleroutput}
  6
  [7,  8,  9]
\end{euleroutput}
\begin{eulercomment}
Indeks juga dapat berupa vektor baris dari indeks. : menunjukkan semua
indeks.
\end{eulercomment}
\begin{eulerprompt}
>v[1:2], A[:,2]
\end{eulerprompt}
\begin{euleroutput}
  [2,  4]
              2 
              5 
              8 
\end{euleroutput}
\begin{eulercomment}
Bentuk singkat untuk : adalah menghilangkan indeks sepenuhnya.
\end{eulercomment}
\begin{eulerprompt}
>A[,2:3]
\end{eulerprompt}
\begin{euleroutput}
              2             3 
              5             6 
              8             9 
\end{euleroutput}
\begin{eulercomment}
Untuk tujuan vektorisasi, elemen-elemen matriks dapat diakses
seolah-olah mereka adalah vektor.
\end{eulercomment}
\begin{eulerprompt}
>A\{4\}
\end{eulerprompt}
\begin{euleroutput}
  4
\end{euleroutput}
\begin{eulercomment}
Sebuah matriks juga dapat diratakan, dengan menggunakan fungsi
redim(). Hal ini diimplementasikan dalam fungsi flatten().
\end{eulercomment}
\begin{eulerprompt}
>redim(A,1,prod(size(A))), flatten(A)
\end{eulerprompt}
\begin{euleroutput}
  [1,  2,  3,  4,  5,  6,  7,  8,  9]
  [1,  2,  3,  4,  5,  6,  7,  8,  9]
\end{euleroutput}
\begin{eulercomment}
Untuk menggunakan matriks pada tabel, mari kita atur ulang ke format
default, dan hitung tabel nilai sinus dan kosinus. Perhatikan bahwa
sudut dinyatakan dalam radian secara default.
\end{eulercomment}
\begin{eulerprompt}
>defformat; w=0°:45°:360°; w=w'; deg(w)
\end{eulerprompt}
\begin{euleroutput}
              0 
             45 
             90 
            135 
            180 
            225 
            270 
            315 
            360 
\end{euleroutput}
\begin{eulercomment}
Sekarang kita menambahkan kolom ke matriks.
\end{eulercomment}
\begin{eulerprompt}
>M = deg(w)|w|cos(w)|sin(w)
\end{eulerprompt}
\begin{euleroutput}
              0             0             1             0 
             45      0.785398      0.707107      0.707107 
             90        1.5708             0             1 
            135       2.35619     -0.707107      0.707107 
            180       3.14159            -1             0 
            225       3.92699     -0.707107     -0.707107 
            270       4.71239             0            -1 
            315       5.49779      0.707107     -0.707107 
            360       6.28319             1             0 
\end{euleroutput}
\begin{eulercomment}
Dengan menggunakan bahasa matriks, kita dapat membuat beberapa tabel
dari beberapa fungsi sekaligus.

Pada contoh berikut, kita menghitung t[j]\textasciicircum{}i untuk i dari 1 hingga n.
Kita mendapatkan sebuah matriks, di mana setiap baris adalah tabel t\textasciicircum{}i
untuk satu i. Dengan kata lain, matriks tersebut memiliki
elemen-elemen lateks: a\_\{i, j\} = t\_j\textasciicircum{}i, \textbackslash{}qad 1 \textbackslash{}le j \textbackslash{}le 101, \textbackslash{}qad 1
\textbackslash{}le i \textbackslash{}le n

Sebuah fungsi yang tidak bekerja untuk input vektor harus
"divektorkan". Hal ini dapat dicapai dengan kata kunci "map" dalam
definisi fungsi. Kemudian fungsi akan dievaluasi untuk setiap elemen
parameter vektor.

Integrasi numerik integrate() hanya bekerja untuk batas interval
skalar. Jadi kita perlu membuat vektornya.
\end{eulercomment}
\begin{eulerprompt}
>function map f(x) := integrate("x^x",1,x)
\end{eulerprompt}
\begin{eulercomment}
Kata kunci "map" membuat vektor fungsi. Fungsi ini sekarang akan
bekerja\\
untuk vektor angka.
\end{eulercomment}
\begin{eulerprompt}
>f([1:5])
\end{eulerprompt}
\begin{euleroutput}
  [0,  2.05045,  13.7251,  113.336,  1241.03]
\end{euleroutput}
\eulerheading{Sub-Matriks dan Elemen Matriks}
\begin{eulercomment}
Untuk mengakses elemen matriks, gunakan notasi kurung.
\end{eulercomment}
\begin{eulerprompt}
>A=[1,2,3;4,5,6;7,8,9], A[2,2]
\end{eulerprompt}
\begin{euleroutput}
              1             2             3 
              4             5             6 
              7             8             9 
  5
\end{euleroutput}
\begin{eulercomment}
Kita dapat mengakses baris lengkap dari sebuah matriks.
\end{eulercomment}
\begin{eulerprompt}
>A[2]
\end{eulerprompt}
\begin{euleroutput}
  [4,  5,  6]
\end{euleroutput}
\begin{eulercomment}
Untuk vektor baris atau kolom, ini mengembalikan elemen vektor.
\end{eulercomment}
\begin{eulerprompt}
>v=1:3; v[2]
\end{eulerprompt}
\begin{euleroutput}
  2
\end{euleroutput}
\begin{eulercomment}
Untuk memastikan, Anda mendapatkan baris pertama untuk matriks 1xn dan
mxn, tentukan semua kolom menggunakan indeks kedua yang kosong.
\end{eulercomment}
\begin{eulerprompt}
>A[2,]
\end{eulerprompt}
\begin{euleroutput}
  [4,  5,  6]
\end{euleroutput}
\begin{eulercomment}
Jika indeks adalah vektor indeks, Euler akan mengembalikan baris-baris
yang sesuai dari matriks.

Di sini kita menginginkan baris pertama dan kedua dari A.
\end{eulercomment}
\begin{eulerprompt}
>A[[1,2]]
\end{eulerprompt}
\begin{euleroutput}
              1             2             3 
              4             5             6 
\end{euleroutput}
\begin{eulercomment}
Kita bahkan dapat menyusun ulang A menggunakan vektor indeks.
Tepatnya, kita tidak mengubah A di sini, tetapi menghitung versi
susunan ulang dari A.
\end{eulercomment}
\begin{eulerprompt}
>A[[3,2,1]]
\end{eulerprompt}
\begin{euleroutput}
              7             8             9 
              4             5             6 
              1             2             3 
\end{euleroutput}
\begin{eulercomment}
Trik indeks juga bekerja dengan kolom.

Contoh ini memilih semua baris A dan kolom kedua dan ketiga.
\end{eulercomment}
\begin{eulerprompt}
>A[1:3,2:3]
\end{eulerprompt}
\begin{euleroutput}
              2             3 
              5             6 
              8             9 
\end{euleroutput}
\begin{eulercomment}
Untuk singkatan ":" menunjukkan semua indeks baris atau kolom.
\end{eulercomment}
\begin{eulerprompt}
>A[:,3]
\end{eulerprompt}
\begin{euleroutput}
              3 
              6 
              9 
\end{euleroutput}
\begin{eulercomment}
Sebagai alternatif, biarkan indeks pertama kosong.
\end{eulercomment}
\begin{eulerprompt}
>A[,2:3]
\end{eulerprompt}
\begin{euleroutput}
              2             3 
              5             6 
              8             9 
\end{euleroutput}
\begin{eulercomment}
Kita juga bisa mendapatkan baris terakhir A.
\end{eulercomment}
\begin{eulerprompt}
>A[-1]
\end{eulerprompt}
\begin{euleroutput}
  [7,  8,  9]
\end{euleroutput}
\begin{eulercomment}
Sekarang mari kita ubah elemen-elemen dari A dengan memberikan sebuah
submatriks dari A ke suatu nilai. Hal ini sebenarnya mengubah matriks
A yang tersimpan.
\end{eulercomment}
\begin{eulerprompt}
>A[1,1]=4
\end{eulerprompt}
\begin{euleroutput}
              4             2             3 
              4             5             6 
              7             8             9 
\end{euleroutput}
\begin{eulercomment}
Kita juga dapat menetapkan nilai pada deretan A.
\end{eulercomment}
\begin{eulerprompt}
>A[1]=[-1,-1,-1]
\end{eulerprompt}
\begin{euleroutput}
             -1            -1            -1 
              4             5             6 
              7             8             9 
\end{euleroutput}
\begin{eulercomment}
Kami bahkan dapat menetapkan ke sub-matriks jika memiliki ukuran yang
tepat.
\end{eulercomment}
\begin{eulerprompt}
>A[1:2,1:2]=[5,6;7,8]
\end{eulerprompt}
\begin{euleroutput}
              5             6            -1 
              7             8             6 
              7             8             9 
\end{euleroutput}
\begin{eulercomment}
Selain itu, beberapa jalan pintas diperbolehkan.
\end{eulercomment}
\begin{eulerprompt}
>A[1:2,1:2]=0
\end{eulerprompt}
\begin{euleroutput}
              0             0            -1 
              0             0             6 
              7             8             9 
\end{euleroutput}
\begin{eulercomment}
Peringatan: Indeks di luar batas akan mengembalikan matriks kosong,
atau pesan kesalahan, tergantung pada pengaturan sistem. Standarnya
adalah pesan kesalahan. Namun, ingatlah bahwa indeks negatif dapat
digunakan untuk mengakses elemen-elemen matriks yang dihitung dari
akhir.
\end{eulercomment}
\begin{eulerprompt}
>A[4]
\end{eulerprompt}
\begin{euleroutput}
  Row index 4 out of bounds!
  Error in:
  A[4] ...
      ^
\end{euleroutput}
\eulerheading{Pengurutan dan Pengacakan}
\begin{eulercomment}
Fungsi mengurutkan() mengurutkan vektor baris.
\end{eulercomment}
\begin{eulerprompt}
> sort([5,6,4,8,1,9])
\end{eulerprompt}
\begin{euleroutput}
  [1,  4,  5,  6,  8,  9]
\end{euleroutput}
\begin{eulercomment}
Sering kali diperlukan untuk mengetahui indeks vektor yang diurutkan
dalam vektor aslinya. Hal ini dapat digunakan untuk menyusun ulang
vektor lain dengan cara yang sama.

Mari kita mengacak sebuah vektor.
\end{eulercomment}
\begin{eulerprompt}
>v=shuffle(1:10)
\end{eulerprompt}
\begin{euleroutput}
  [4,  5,  10,  6,  8,  9,  1,  7,  2,  3]
\end{euleroutput}
\begin{eulercomment}
Indeks berisi urutan v yang tepat.
\end{eulercomment}
\begin{eulerprompt}
>\{vs,ind\}=sort(v); v[ind]
\end{eulerprompt}
\begin{euleroutput}
  [1,  2,  3,  4,  5,  6,  7,  8,  9,  10]
\end{euleroutput}
\begin{eulercomment}
Hal ini juga berlaku untuk vektor string.
\end{eulercomment}
\begin{eulerprompt}
>s=["a","d","e","a","aa","e"]
\end{eulerprompt}
\begin{euleroutput}
  a
  d
  e
  a
  aa
  e
\end{euleroutput}
\begin{eulerprompt}
>\{ss,ind\}=sort(s); ss
\end{eulerprompt}
\begin{euleroutput}
  a
  a
  aa
  d
  e
  e
\end{euleroutput}
\begin{eulercomment}
Seperti yang Anda lihat, posisi entri ganda agak acak.
\end{eulercomment}
\begin{eulerprompt}
>ind
\end{eulerprompt}
\begin{euleroutput}
  [4,  1,  5,  2,  6,  3]
\end{euleroutput}
\begin{eulercomment}
Fungsi unique mengembalikan daftar terurut dari elemen unik sebuah
vektor.
\end{eulercomment}
\begin{eulerprompt}
>intrandom(1,10,10), unique(%)
\end{eulerprompt}
\begin{euleroutput}
  [4,  4,  9,  2,  6,  5,  10,  6,  5,  1]
  [1,  2,  4,  5,  6,  9,  10]
\end{euleroutput}
\begin{eulercomment}
Hal ini juga berlaku untuk vektor string.
\end{eulercomment}
\begin{eulerprompt}
>unique(s)
\end{eulerprompt}
\begin{euleroutput}
  a
  aa
  d
  e
\end{euleroutput}
\eulerheading{Aljabar Linier}
\begin{eulercomment}
EMT memiliki banyak fungsi untuk menyelesaikan sistem linier, sistem
jarang, atau masalah regresi.

Untuk sistem linier Ax=b, Anda dapat menggunakan algoritma Gauss,
matriks invers, atau kecocokan linier. Operator A\textbackslash{}b menggunakan versi
algoritma Gauss.
\end{eulercomment}
\begin{eulerprompt}
>A=[1,2;3,4]; b=[5;6]; A\(\backslash\)b
\end{eulerprompt}
\begin{euleroutput}
             -4 
            4.5 
\end{euleroutput}
\begin{eulercomment}
Sebagai contoh lain, kita membuat matriks 200x200 dan jumlah barisnya.
Kemudian kita selesaikan Ax = b dengan menggunakan matriks
kebalikannya. Kita mengukur kesalahan sebagai deviasi maksimal dari
semua elemen dari 1, yang tentu saja merupakan solusi yang benar.
\end{eulercomment}
\begin{eulerprompt}
>A=normal(200,200); b=sum(A); longest totalmax(abs(inv(A).b-1))
\end{eulerprompt}
\begin{euleroutput}
    8.790745908981989e-13 
\end{euleroutput}
\begin{eulercomment}
Jika sistem tidak memiliki solusi, kecocokan linier meminimalkan norma
kesalahan Ax-b.
\end{eulercomment}
\begin{eulerprompt}
>A=[1,2,3;4,5,6;7,8,9]
\end{eulerprompt}
\begin{euleroutput}
              1             2             3 
              4             5             6 
              7             8             9 
\end{euleroutput}
\begin{eulercomment}
Determinan dari matriks ini adalah 0.
\end{eulercomment}
\begin{eulerprompt}
>det(A)
\end{eulerprompt}
\begin{euleroutput}
  0
\end{euleroutput}
\begin{eulercomment}
== Sisipan soal dari PDF ==

1)
\end{eulercomment}
\begin{eulerprompt}
>A=[2,-1;1,4]; b=[7;-5]; A/b
\end{eulerprompt}
\begin{euleroutput}
       0.285714     -0.142857 
           -0.2          -0.8 
\end{euleroutput}
\begin{eulercomment}
2)
\end{eulercomment}
\eulerheading{Matriks Simbolik}
\begin{eulercomment}
Maxima memiliki matriks simbolik. Tentu saja, Maxima dapat digunakan
untuk masalah aljabar linier sederhana. Kita bisa mendefinisikan
matriks untuk Euler dan Maxima dengan \&:=, dan kemudian menggunakannya
dalam ekspresi simbolik. Bentuk [...] yang biasa untuk mendefinisikan
matriks dapat digunakan dalam Euler untuk mendefinisikan matriks
simbolik.
\end{eulercomment}
\begin{eulerprompt}
>A &= [a,1,1;1,a,1;1,1,a]; $A
\end{eulerprompt}
\begin{eulerformula}
\[
\begin{pmatrix}a & 1 & 1 \\ 1 & a & 1 \\ 1 & 1 & a \\ \end{pmatrix}
\]
\end{eulerformula}
\begin{eulerprompt}
>$&det(A), $&factor(%)
\end{eulerprompt}
\begin{eulerformula}
\[
\left(a-1\right)^2\,\left(a+2\right)
\]
\end{eulerformula}
\eulerimg{0}{images/EMT4aljabar_Pradika Larasati_23030630003-094-large.png}
\begin{eulerprompt}
>$&invert(A) with a=0
\end{eulerprompt}
\begin{eulerformula}
\[
\begin{pmatrix}-\frac{1}{2} & \frac{1}{2} & \frac{1}{2} \\ \frac{1  }{2} & -\frac{1}{2} & \frac{1}{2} \\ \frac{1}{2} & \frac{1}{2} & -  \frac{1}{2} \\ \end{pmatrix}
\]
\end{eulerformula}
\begin{eulerprompt}
>A &= [1,a;b,2]; $A
\end{eulerprompt}
\begin{eulerformula}
\[
\begin{pmatrix}1 & a \\ b & 2 \\ \end{pmatrix}
\]
\end{eulerformula}
\begin{eulercomment}
Seperti semua variabel simbolik, matriks ini dapat digunakan dalam
ekspresi simbolik lainnya.
\end{eulercomment}
\begin{eulerprompt}
>$&det(A-x*ident(2)), $&solve(%,x)
\end{eulerprompt}
\begin{eulercomment}
Nilai eigen juga dapat dihitung secara otomatis. Hasilnya adalah
sebuah vektor dengan dua vektor nilai eigen dan kelipatannya.
\end{eulercomment}
\begin{eulerprompt}
>$&eigenvalues([a,1;1,a])
\end{eulerprompt}
\begin{eulercomment}
Untuk mengekstrak vektor eigen tertentu, diperlukan pengindeksan yang
cermat.
\end{eulercomment}
\begin{eulerprompt}
>$&eigenvectors([a,1;1,a]), &%[2][1][1]
\end{eulerprompt}
\begin{euleroutput}
  
                                 [1, - 1]
  
\end{euleroutput}
\begin{eulercomment}
Matriks simbolik dapat dievaluasi dalam Euler secara numerik seperti
halnya ekspresi simbolik lainnya.
\end{eulercomment}
\begin{eulerprompt}
>A(a=4,b=5)
\end{eulerprompt}
\begin{euleroutput}
              1             4 
              5             2 
\end{euleroutput}
\begin{eulercomment}
Dalam ekspresi simbolis, gunakan dengan.
\end{eulercomment}
\begin{eulerprompt}
>$&A with [a=4,b=5]
\end{eulerprompt}
\begin{eulerformula}
\[
\begin{pmatrix}1 & 4 \\ 5 & 2 \\ \end{pmatrix}
\]
\end{eulerformula}
\begin{eulercomment}
Akses ke baris matriks simbolik bekerja seperti halnya matriks
numerik.
\end{eulercomment}
\begin{eulerprompt}
>$&A[1]
\end{eulerprompt}
\begin{eulercomment}
Ekspresi simbolik dapat berisi sebuah penugasan. Dan itu mengubah
matriks A.
\end{eulercomment}
\begin{eulerprompt}
>&A[1,1]:=t+1; $&A
\end{eulerprompt}
\begin{eulerformula}
\[
\begin{pmatrix}t+1 & a \\ b & 2 \\ \end{pmatrix}
\]
\end{eulerformula}
\begin{eulercomment}
Terdapat fungsi-fungsi simbolik dalam Maxima untuk membuat vektor dan
matriks. Untuk hal ini, lihat dokumentasi Maxima atau tutorial tentang
Maxima di EMT.
\end{eulercomment}
\begin{eulerprompt}
>v &= makelist(1/(i+j),i,1,3); $v
\end{eulerprompt}
\begin{eulerformula}
\[
\left[ \frac{1}{j+1} , \frac{1}{j+2} , \frac{1}{j+3} \right] 
\]
\end{eulerformula}
\begin{eulerttcomment}
 
\end{eulerttcomment}
\begin{eulerprompt}
>B &:= [1,2;3,4]; $B, $&invert(B)
\end{eulerprompt}
\begin{eulerformula}
\[
\begin{pmatrix}-2 & 1 \\ \frac{3}{2} & -\frac{1}{2} \\   \end{pmatrix}
\]
\end{eulerformula}
\eulerimg{1}{images/EMT4aljabar_Pradika Larasati_23030630003-101-large.png}
\begin{eulercomment}
Hasilnya dapat dievaluasi secara numerik dalam Euler. Untuk informasi
lebih lanjut tentang Maxima, lihat pengantar Maxima.
\end{eulercomment}
\begin{eulerprompt}
>$&invert(B)()
\end{eulerprompt}
\begin{euleroutput}
             -2             1 
            1.5          -0.5 
\end{euleroutput}
\begin{eulercomment}
Euler juga memiliki sebuah fungsi yang kuat xinv(), yang melakukan
usaha yang lebih besar dan mendapatkan hasil yang lebih tepat.

Perhatikan, bahwa dengan \&:= matriks B telah didefinisikan sebagai
simbolik dalam ekspresi simbolik dan sebagai numerik dalam ekspresi
numerik. Jadi kita dapat menggunakannya di sini.
\end{eulercomment}
\begin{eulerprompt}
>longest B.xinv(B)
\end{eulerprompt}
\begin{euleroutput}
                        1                       0 
                        0                       1 
\end{euleroutput}
\begin{eulercomment}
Misalnya, nilai eigen dari A dapat dihitung secara numerik.
\end{eulercomment}
\begin{eulerprompt}
>A=[1,2,3;4,5,6;7,8,9]; real(eigenvalues(A))
\end{eulerprompt}
\begin{euleroutput}
  [16.1168,  -1.11684,  0]
\end{euleroutput}
\begin{eulercomment}
Atau secara simbolis. Lihat tutorial tentang Maxima untuk detailnya.
\end{eulercomment}
\begin{eulerprompt}
>$&eigenvalues(@A)
\end{eulerprompt}
\eulerheading{Nilai Numerik dalam Ekspresi simbolik}
\begin{eulercomment}
Ekspresi simbolik hanyalah sebuah string yang berisi ekspresi. Jika
kita ingin mendefinisikan nilai baik untuk ekspresi simbolik maupun
ekspresi numerik, kita harus menggunakan "\&:=".
\end{eulercomment}
\begin{eulerprompt}
>A &:= [1,pi;4,5]
\end{eulerprompt}
\begin{euleroutput}
              1       3.14159 
              4             5 
\end{euleroutput}
\begin{eulercomment}
Masih ada perbedaan antara bentuk numerik dan bentuk simbolik. Ketika
mentransfer matriks ke bentuk simbolik, perkiraan pecahan untuk
bilangan real akan digunakan.
\end{eulercomment}
\begin{eulerprompt}
>$&A
\end{eulerprompt}
\begin{eulercomment}
Untuk menghindari hal ini, ada fungsi "mxmset(variable)".
\end{eulercomment}
\begin{eulerprompt}
>mxmset(A); $&A
\end{eulerprompt}
\begin{eulercomment}
Maxima juga dapat menghitung dengan angka floating point, dan bahkan
dengan angka mengambang yang besar dengan 32 digit. Namun, evaluasinya
jauh lebih lambat.
\end{eulercomment}
\begin{eulerprompt}
>$&bfloat(sqrt(2)), $&float(sqrt(2))
\end{eulerprompt}
\begin{eulerformula}
\[
1.414213562373095
\]
\end{eulerformula}
\eulerimg{0}{images/EMT4aljabar_Pradika Larasati_23030630003-103-large.png}
\begin{eulercomment}
Ketepatan angka floating point yang besar dapat diubah.
\end{eulercomment}
\begin{eulerprompt}
>&fpprec:=100; &bfloat(pi)
\end{eulerprompt}
\begin{euleroutput}
  
          3.14159265358979323846264338327950288419716939937510582097494\(\backslash\)
  4592307816406286208998628034825342117068b0
  
\end{euleroutput}
\begin{eulercomment}
Variabel numerik dapat digunakan dalam ekspresi simbolik apa pun
dengan menggunakan "@var".

Perhatikan bahwa ini hanya diperlukan, jika variabel telah
didefinisikan dengan ":=" atau "=" sebagai variabel numerik.
\end{eulercomment}
\begin{eulerprompt}
>B:=[1,pi;3,4]; $&det(@B)
\end{eulerprompt}
\begin{eulerformula}
\[
-5.424777960769379
\]
\end{eulerformula}
\begin{eulercomment}
\begin{eulercomment}
\eulerheading{Demo - Suku Bunga}
\begin{eulercomment}
Di bawah ini, kami menggunakan Euler Math Toolbox (EMT) untuk
menghitung suku bunga. Kami melakukannya secara numerik dan simbolis
untuk menunjukkan kepada Anda bagaimana Euler dapat digunakan untuk
memecahkan masalah kehidupan nyata.

Asumsikan Anda memiliki modal awal sebesar 5000 (katakanlah dalam
dolar).
\end{eulercomment}
\begin{eulerprompt}
>K=5000
\end{eulerprompt}
\begin{euleroutput}
  5000
\end{euleroutput}
\begin{eulercomment}
Sekarang kita asumsikan suku bunga 3\% per tahun. Mari kita tambahkan
satu suku bunga sederhana dan hitung hasilnya.
\end{eulercomment}
\begin{eulerprompt}
>K*1.03
\end{eulerprompt}
\begin{euleroutput}
  5150
\end{euleroutput}
\begin{eulercomment}
Euler juga akan memahami sintaks berikut ini.
\end{eulercomment}
\begin{eulerprompt}
>K+K*3%
\end{eulerprompt}
\begin{euleroutput}
  5150
\end{euleroutput}
\begin{eulercomment}
Tetapi lebih mudah untuk menggunakan faktor
\end{eulercomment}
\begin{eulerprompt}
>q=1+3%, K*q
\end{eulerprompt}
\begin{euleroutput}
  1.03
  5150
\end{euleroutput}
\begin{eulercomment}
Untuk 10 tahun, kita cukup mengalikan faktor-faktor tersebut dan
mendapatkan nilai akhir dengan suku bunga majemuk.
\end{eulercomment}
\begin{eulerprompt}
>K*q^10
\end{eulerprompt}
\begin{euleroutput}
  6719.58189672
\end{euleroutput}
\begin{eulercomment}
Untuk tujuan kita, kita bisa menetapkan formatnya menjadi 2 digit
setelah titik desimal.
\end{eulercomment}
\begin{eulerprompt}
>format(12,2); K*q^10
\end{eulerprompt}
\begin{euleroutput}
      6719.58 
\end{euleroutput}
\begin{eulercomment}
Mari kita cetak angka yang dibulatkan menjadi 2 digit dalam kalimat
lengkap.
\end{eulercomment}
\begin{eulerprompt}
>"Starting from " + K + "$ you get " + round(K*q^10,2) + "$."
\end{eulerprompt}
\begin{euleroutput}
  Starting from 5000$ you get 6719.58$.
\end{euleroutput}
\begin{eulercomment}
Bagaimana jika kita ingin mengetahui hasil antara dari tahun ke-1
hingga tahun ke-9? Untuk hal ini, bahasa matriks Euler sangat
membantu. Anda tidak perlu menulis perulangan, tetapi cukup masukkan
\end{eulercomment}
\begin{eulerprompt}
>K*q^(0:10)
\end{eulerprompt}
\begin{euleroutput}
  Real 1 x 11 matrix
  
      5000.00     5150.00     5304.50     5463.64     ...
\end{euleroutput}
\begin{eulercomment}
Bagaimana keajaiban ini bekerja? Pertama, ekspresi 0:10 mengembalikan
sebuah vektor bilangan bulat.
\end{eulercomment}
\begin{eulerprompt}
>short 0:10
\end{eulerprompt}
\begin{euleroutput}
  [0,  1,  2,  3,  4,  5,  6,  7,  8,  9,  10]
\end{euleroutput}
\begin{eulercomment}
Kemudian semua operator dan fungsi dalam Euler dapat diterapkan pada
vektor elemen demi elemen. Jadi
\end{eulercomment}
\begin{eulerprompt}
>short q^(0:10)
\end{eulerprompt}
\begin{euleroutput}
  [1,  1.03,  1.0609,  1.0927,  1.1255,  1.1593,  1.1941,  1.2299,
  1.2668,  1.3048,  1.3439]
\end{euleroutput}
\begin{eulercomment}
adalah vektor faktor q\textasciicircum{}0 hingga q\textasciicircum{}10. Ini dikalikan dengan K, dan kita
mendapatkan vektor nilai.
\end{eulercomment}
\begin{eulerprompt}
>VK=K*q^(0:10);
\end{eulerprompt}
\begin{euleroutput}
  Commands must be separated by semicolon or comma!
  Found: K*q^(0:10) (character 75)
  You can disable this in the Options menu.
  Error in:
  VK=:K*q^(0:10) ...
      ^
\end{euleroutput}
\begin{eulercomment}
Tentu saja, cara realistis untuk menghitung suku bunga ini adalah
dengan membulatkan ke sen terdekat setiap tahunnya. Mari kita
tambahkan fungsi untuk ini.
\end{eulercomment}
\begin{eulerprompt}
>function oneyear (K) := round(K*q,2)
\end{eulerprompt}
\begin{eulercomment}
Mari kita bandingkan kedua hasil tersebut, dengan dan tanpa
pembulatan.
\end{eulercomment}
\begin{eulerprompt}
>longest oneyear(1234.57), longest 1234.57*q
\end{eulerprompt}
\begin{euleroutput}
                  1271.61 
                1271.6071 
\end{euleroutput}
\begin{eulercomment}
Sekarang tidak ada rumus sederhana untuk tahun ke-n, dan kita harus
mengulang selama bertahun-tahun. Euler menyediakan banyak solusi untuk
ini.

Cara termudah adalah iterasi fungsi, yang mengulang fungsi yang
diberikan beberapa kali.
\end{eulercomment}
\begin{eulerprompt}
>VKr=iterate("oneyear",5000,10)
\end{eulerprompt}
\begin{euleroutput}
  Variable q not found!
  Use global or local variables defined in function oneyear.
  oneyear:
      useglobal; return round(K*q,2) 
  niterate:
      x=f$(x,args());
  Try "trace errors" to inspect local variables after errors.
  iterate:
      return niterate(f$,x,n,till;args());
\end{euleroutput}
\begin{eulercomment}
Kita bisa mencetaknya dengan cara yang bersahabat, menggunakan format
kami dengan angka desimal yang tetap.
\end{eulercomment}
\begin{eulerprompt}
>VKr'
\end{eulerprompt}
\begin{euleroutput}
      5000.00 
      5150.00 
      5304.50 
      5463.64 
      5627.55 
      5796.38 
      5970.27 
      6149.38 
      6333.86 
      6523.88 
      6719.60 
\end{euleroutput}
\begin{eulercomment}
Untuk mendapatkan elemen vektor tertentu, kita menggunakan indeks
dalam tanda kurung siku.
\end{eulercomment}
\begin{eulerprompt}
>VKr[2], VKr[1:3]
\end{eulerprompt}
\begin{euleroutput}
      5150.00 
      5000.00     5150.00     5304.50 
\end{euleroutput}
\begin{eulercomment}
Yang mengejutkan, kita juga dapat menggunakan vektor indeks. Ingatlah
bahwa 1:3 menghasilkan vektor [1,2,3].

Mari kita bandingkan elemen terakhir dari nilai yang dibulatkan dengan
nilai penuh.
\end{eulercomment}
\begin{eulerprompt}
>VKr[-1], VK[-1]
\end{eulerprompt}
\begin{euleroutput}
      6719.60 
      6719.58 
\end{euleroutput}
\begin{eulercomment}
Perbedaannya sangat kecil.

\begin{eulercomment}
\eulerheading{Menyelesaikan Persamaan}
\begin{eulercomment}
Sekarang kita ambil fungsi yang lebih maju, yang menambahkan tingkat
uang tertentu setiap tahun.
\end{eulercomment}
\begin{eulerprompt}
>function onepay (K) := K*q+R
\end{eulerprompt}
\begin{eulercomment}
Kita tidak perlu menentukan q atau R untuk definisi fungsi. Hanya jika
kita menjalankan perintah, kita harus mendefinisikan nilai-nilai ini.
Kami memilih R = 200.
\end{eulercomment}
\begin{eulerprompt}
>q=200; iterate("onepay",5000,10)
\end{eulerprompt}
\begin{euleroutput}
  [5000,  1.0002e+06,  2.0004e+08,  4.0008e+10,  8.00161e+12,
  1.60032e+15,  3.20064e+17,  6.40129e+19,  1.28026e+22,  2.56051e+24,
  5.12103e+26]
\end{euleroutput}
\begin{eulercomment}
Bagaimana jika kita menghapus jumlah yang sama setiap tahun?
\end{eulercomment}
\begin{eulerprompt}
>q=-200; iterate("onepay",5000,10)
\end{eulerprompt}
\begin{euleroutput}
  [5000,  -999800,  1.9996e+08,  -3.9992e+10,  7.99841e+12,
  -1.59968e+15,  3.19936e+17,  -6.39873e+19,  1.27975e+22,
  -2.55949e+24,  5.11898e+26]
\end{euleroutput}
\begin{eulercomment}
Kami melihat bahwa uangnya berkurang. Jelas, jika kita hanya
mendapatkan 150 bunga di tahun pertama, tetapi menghapus 200, kita
kehilangan uang setiap tahun.

Bagaimana kita dapat menentukan berapa tahun uang itu akan bertahan?
Kita harus menulis perulangan untuk ini. Cara termudah adalah dengan
melakukan perulangan yang cukup lama.
\end{eulercomment}
\begin{eulerprompt}
>VKq=iterate("onepay",5000,50)
\end{eulerprompt}
\begin{euleroutput}
  [5000,  -999800,  1.9996e+08,  -3.9992e+10,  7.99841e+12,
  -1.59968e+15,  3.19936e+17,  -6.39873e+19,  1.27975e+22,
  -2.55949e+24,  5.11898e+26,  -1.0238e+29,  2.04759e+31,  -4.09518e+33,
  8.19037e+35,  -1.63807e+38,  3.27615e+40,  -6.5523e+42,  1.31046e+45,
  -2.62092e+47,  5.24184e+49,  -1.04837e+52,  2.09673e+54,
  -4.19347e+56,  8.38694e+58,  -1.67739e+61,  3.35478e+63,
  -6.70955e+65,  1.34191e+68,  -2.68382e+70,  5.36764e+72,
  -1.07353e+75,  2.14706e+77,  -4.29411e+79,  8.58823e+81,
  -1.71765e+84,  3.43529e+86,  -6.87058e+88,  1.37412e+91,
  -2.74823e+93,  5.49646e+95,  -1.09929e+98,  2.19859e+100,
  -4.39717e+102,  8.79434e+104,  -1.75887e+107,  3.51774e+109,
  -7.03547e+111,  1.40709e+114,  -2.81419e+116,  5.62838e+118]
\end{euleroutput}
\begin{eulercomment}
Dengan menggunakan bahasa matriks, kita dapat menentukan nilai negatif
pertama dengan cara berikut.
\end{eulercomment}
\begin{eulerprompt}
>min(nonzeros(VKq<0))
\end{eulerprompt}
\begin{euleroutput}
  2
\end{euleroutput}
\begin{eulercomment}
Alasannya adalah karena nonzeros(VKR\textless{}0) mengembalikan vektor dengan
indeks i, di mana VKR[i]\textless{}0, dan min menghitung indeks minimal.

Karena vektor selalu dimulai dengan indeks 1, maka jawabannya adalah
47 tahun.

Fungsi iterate() memiliki satu trik lagi. Fungsi ini dapat menerima
kondisi akhir sebagai argumen. Kemudian fungsi ini akan mengembalikan
nilai dan jumlah iterasi.
\end{eulercomment}
\begin{eulerprompt}
>\{x,n\}=iterate("onepay",5000,till="x<0"); x, n,
\end{eulerprompt}
\begin{euleroutput}
  -999800
  1
\end{euleroutput}
\begin{eulercomment}
Mari kita coba menjawab pertanyaan yang lebih ambigu. Anggaplah kita
tahu bahwa nilainya adalah 0 setelah 50 tahun. Berapakah tingkat suku
bunganya?

Ini adalah pertanyaan yang hanya bisa dijawab secara numerik. Di bawah
ini, kami akan menurunkan rumus yang diperlukan. Kemudian Anda akan
melihat bahwa tidak ada rumus yang mudah untuk suku bunga. Namun untuk
saat ini, kita akan mencari solusi numerik.

Langkah pertama adalah mendefinisikan sebuah fungsi yang melakukan
iterasi sebanyak n kali. Kita tambahkan semua parameter ke fungsi ini.
\end{eulercomment}
\begin{eulerprompt}
>function f(K,R,P,n) := iterate("x*(1+P/100)+R",K,n;P,R)[-1]
\end{eulerprompt}
\begin{eulercomment}
Perulangannya sama seperti di atas

\end{eulercomment}
\begin{eulerformula}
\[
x_{n+1} = x_n \cdot \left(1+ \frac{P}{100}\right) + R
\]
\end{eulerformula}
\begin{eulercomment}
Tetapi kita tidak lagi menggunakan nilai global R dalam ekspresi kita.
Fungsi-fungsi seperti iterate() memiliki trik khusus dalam Euler. Anda
bisa mengoper nilai variabel dalam ekspresi sebagai parameter titik
koma. Dalam hal ini P dan R.

Selain itu, kita hanya tertarik pada nilai terakhir. Jadi kita
mengambil indeks [-1].

Mari kita coba sebuah tes.
\end{eulercomment}
\begin{eulerprompt}
>f(5000,-200,3,47)
\end{eulerprompt}
\begin{euleroutput}
  Function f not found.
  Try list ... to find functions!
  Error in:
  f(5000,-200,3,47) ...
                   ^
\end{euleroutput}
\begin{eulercomment}
Sekarang kita bisa menyelesaikan masalah kita.
\end{eulercomment}
\begin{eulerprompt}
>solve("f(5000,-200,x,50)",3)
\end{eulerprompt}
\begin{euleroutput}
         3.15 
\end{euleroutput}
\begin{eulercomment}
Rutin penyelesaian menyelesaikan ekspresi = 0 untuk variabel x.
Jawabannya adalah 3,15\% per tahun. Kita mengambil nilai awal 3\% untuk
algoritma ini. Fungsi solve() selalu membutuhkan nilai awal.

Kita dapat menggunakan fungsi yang sama untuk menyelesaikan pertanyaan
berikut: Berapa banyak yang dapat kita hapus per tahun sehingga modal
awal habis setelah 20 tahun dengan asumsi tingkat bunga 3\% per tahun.
\end{eulercomment}
\begin{eulerprompt}
>solve("f(5000,x,3,20)",-200)
\end{eulerprompt}
\begin{euleroutput}
      -336.08 
\end{euleroutput}
\begin{eulercomment}
Perhatikan bahwa Anda tidak dapat menyelesaikan jumlah tahun, karena
fungsi kita mengasumsikan n sebagai nilai bilangan bulat.

\end{eulercomment}
\eulersubheading{Solusi Simbolik untuk Masalah Suku Bunga}
\begin{eulercomment}
Kita dapat menggunakan bagian simbolik dari Euler untuk mempelajari
masalah ini. Pertama, kita mendefinisikan fungsi onepay() secara
simbolik.
\end{eulercomment}
\begin{eulerprompt}
>function op(K) &= K*q+R; $&op(K)
\end{eulerprompt}
\begin{eulerformula}
\[
R+q\,K
\]
\end{eulerformula}
\begin{eulercomment}
Sekarang kita bisa mengulangi hal ini.
\end{eulercomment}
\begin{eulerprompt}
>$&op(op(op(op(K)))), $&expand(%)
\end{eulerprompt}
\begin{eulerformula}
\[
q^3\,R+q^2\,R+q\,R+R+q^4\,K
\]
\end{eulerformula}
\eulerimg{0}{images/EMT4aljabar_Pradika Larasati_23030630003-108-large.png}
\begin{eulercomment}
Kita melihat sebuah pola. Setelah n periode, kita memiliki

lateks: K\_n = q\textasciicircum{}n K + R (1+q+\textbackslash{} titik-titik+q\textasciicircum{}\{n-1\}) = q\textasciicircum{}n K +
\textbackslash{}frac\{q\textasciicircum{}n-1\}\{q-1\} R

Rumus tersebut adalah rumus untuk jumlah geometris, yang dikenal
dengan Maxima.
\end{eulercomment}
\begin{eulerprompt}
>&sum(q^k,k,0,n-1); $& % = ev(%,simpsum)
\end{eulerprompt}
\begin{eulerformula}
\[
\sum_{k=0}^{n-1}{q^{k}}=\frac{q^{n}-1}{q-1}
\]
\end{eulerformula}
\begin{eulercomment}
Ini sedikit rumit. Penjumlahan dievaluasi dengan flag "simpsum" untuk
menguranginya menjadi hasil bagi.

Mari kita buat sebuah fungsi untuk ini.
\end{eulercomment}
\begin{eulerprompt}
>function fs(K,R,P,n) &= (1+P/100)^n*K + ((1+P/100)^n-1)/(P/100)*R; $&fs(K,R,P,n)
\end{eulerprompt}
\begin{eulerformula}
\[
\frac{100\,\left(\left(\frac{P}{100}+1\right)^{n}-1\right)\,R}{P}+K  \,\left(\frac{P}{100}+1\right)^{n}
\]
\end{eulerformula}
\begin{eulercomment}
Fungsi ini melakukan hal yang sama seperti fungsi f kita sebelumnya.
Tetapi fungsi ini lebih efektif.
\end{eulercomment}
\begin{eulerprompt}
>longest f(5000,-200,3,47), longest fs(5000,-200,3,47)
\end{eulerprompt}
\begin{euleroutput}
       -19.82504734650985 
       -19.82504734652684 
\end{euleroutput}
\begin{eulercomment}
Sekarang kita dapat menggunakannya untuk menanyakan waktu n. Kapan
modal kita habis? Perkiraan awal kita adalah 30 tahun.
\end{eulercomment}
\begin{eulerprompt}
>solve("fs(5000,-330,3,x)",30)
\end{eulerprompt}
\begin{euleroutput}
        20.51 
\end{euleroutput}
\begin{eulercomment}
Jawaban ini mengatakan bahwa nilai tersebut akan menjadi negatif
setelah 21 tahun.

Kita juga bisa menggunakan sisi simbolis dari Euler untuk menghitung
rumus pembayaran.

Asumsikan kita mendapatkan pinjaman sebesar K, dan membayar n kali
pembayaran sebesar R (dimulai setelah tahun pertama) sehingga
menyisakan sisa utang sebesar Kn (pada saat pembayaran terakhir).
Rumus untuk hal ini adalah sebagai berikut
\end{eulercomment}
\begin{eulerprompt}
>equ &= fs(K,R,P,n)=Kn; $&equ
\end{eulerprompt}
\begin{eulercomment}
Biasanya rumus ini diberikan dalam bentuk

lateks: i = \textbackslash{}frac\{P\}\{100\}
\end{eulercomment}
\begin{eulerprompt}
>equ &= (equ with P=100*i); $&equ
\end{eulerprompt}
\begin{eulercomment}
Kita dapat menyelesaikan laju R secara simbolis.
\end{eulercomment}
\begin{eulerprompt}
>$&solve(equ,R)
\end{eulerprompt}
\begin{eulercomment}
Seperti yang dapat Anda lihat dari rumusnya, fungsi ini mengembalikan
kesalahan floating point untuk i = 0. Euler tetap memplotnya.

Tentu saja, kita memiliki batas berikut.
\end{eulercomment}
\begin{eulerprompt}
>$&limit(R(5000,0,x,10),x,0)
\end{eulerprompt}
\begin{eulercomment}
Jelasnya, tanpa bunga kita harus membayar kembali 10 suku bunga 500.

Persamaan ini juga dapat diselesaikan untuk n. Akan terlihat lebih
baik jika kita menerapkan beberapa penyederhanaan.
\end{eulercomment}
\begin{eulerprompt}
>fn &= solve(equ,n) | ratsimp; $&fn
\end{eulerprompt}
\begin{eulerformula}
\[
\left[  \right] 
\]
\end{eulerformula}
\begin{eulercomment}
== Sisipan Soal dari PDF ==

1)James deposits \textdollar{}250 in a retirement account each month beginning at
age 40. If the investment earns 5\% interest, compounded monthly, how
much will have accumulated in the account when he retires 27 years
later?
\end{eulercomment}
\begin{eulerprompt}
>P=250; r=5%; n=12; t=27; P*((1+(r/n))^(n*t))
\end{eulerprompt}
\begin{euleroutput}
  961.655430384
\end{euleroutput}
\begin{eulercomment}
2) The cost of a house is \textdollar{}98,000. The down payment is \textdollar{}16,000, the
interest rate is and the loan period is 25 years. What is the monthly
mortgage payment?
\end{eulercomment}
\begin{eulerprompt}
>P=135-18; r=6.5% ; n=20*12; P*((r*(1+r)^n)/((1+r)^n-1))
\end{eulerprompt}
\begin{euleroutput}
  7.60500207584
\end{euleroutput}
\end{eulernotebook}
\end{document}
